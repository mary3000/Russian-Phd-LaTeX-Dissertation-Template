%&preformat-disser
\RequirePackage[l2tabu,orthodox]{nag} % Раскомментировав, можно в логе получать рекомендации относительно правильного использования пакетов и предупреждения об устаревших и нерекомендуемых пакетах
% Формат А4, 14pt (ГОСТ Р 7.0.11-2011, 5.3.6)
\documentclass[a4paper,14pt,oneside,openany,oldfontcommands]{memoir}

\input{common/setup}            % общие настройки шаблона
\input{common/packages}         % Пакеты общие для диссертации и автореферата
\synopsisfalse                      % Этот документ --- не автореферат
\input{Dissertation/dispackages}    % Пакеты для диссертации
\input{Dissertation/userpackages}   % Пакеты для специфических пользовательских задач

\input{Dissertation/setup}      % Упрощённые настройки шаблона

\input{common/newnames}         % Новые переменные, для всего проекта

\input{common/data}             % Основные сведения
\input{common/fonts}            % Определение шрифтов (частичное)
\input{common/styles}           % Стили общие для диссертации и автореферата
\input{Dissertation/disstyles}  % Стили для диссертации
\input{Dissertation/userstyles} % Стили для специфических пользовательских задач

%%% Библиография. Выбор движка для реализации %%%
% Здесь только проверка установленного ключа. Сама настройка выбора движка
% размещена в common/setup.tex
\ifnumequal{\value{bibliosel}}{0}{%
    \input{biblio/predefined}   % Встроенная реализация с загрузкой файла через движок bibtex8
}{
    \input{biblio/biblatex}     % Реализация пакетом biblatex через движок biber
}

% Вывести информацию о выбранных опциях в лог сборки
\typeout{Selected options:}
\typeout{Draft mode: \arabic{draft}}
\typeout{Font: \arabic{fontfamily}}
\typeout{AltFont: \arabic{usealtfont}}
\typeout{Bibliography backend: \arabic{bibliosel}}
\typeout{Precompile images: \arabic{imgprecompile}}
% Вывести информацию о версиях используемых библиотек в лог сборки
\listfiles

%%% Управление компиляцией отдельных частей диссертации %%%
% Необходимо сначала иметь полностью скомпилированный документ, чтобы все
% промежуточные файлы были в наличии
% Затем, для вывода отдельных частей можно воспользоваться командой \includeonly
% Ниже примеры использования команды:
%
%\includeonly{Dissertation/part2}
%\includeonly{Dissertation/contents,Dissertation/appendix,Dissertation/conclusion}
%
% Если все команды закомментированы, то документ будет выведен в PDF файл полностью

\usepackage{mdframed}

\usepackage{minted}

%\surroundwithmdframed{minted}

%\RecustomVerbatimEnvironment{Verbatim}{BVerbatim}{}

%\renewcommand{\figurename}{Listing}

\newfontfamily\codeubuntu{Ubuntu Mono}[NFSSFamily=CodeUbuntu]

\usemintedstyle{vs}
\setminted[]{tabsize=2, fontfamily=CodeUbuntu, fontsize=\small, frame=lines, linenos, breaklines}
%\setmintedinline[c++]{style=bw}
\setmintedinline[]{fontsize=auto}

\renewcommand{\textfraction}{0.05} 

\usepackage{float}

\makeatletter
\renewcommand{\fps@figure}{H}
\renewcommand{\fps@table}{H}
\renewcommand{\fps@listing}{H}
\makeatother


\usepackage{xcolor}

\usepackage{multicol}

\usepackage{graphicx}

\usepackage{changepage}

\usepackage{caption}

\lstset{
	basicstyle=\fontsize{5}{5}\selectfont\ttfamily,
	escapeinside={<@}{@>},
	numbers=none,
    frame=single,
}

\usepackage{fancyvrb}

\makeatletter
\newcommand{\iftoggleverb}[1]{%
  \ifcsdef{etb@tgl@#1}
    {\csname etb@tgl@#1\endcsname\iftrue\iffalse}
    {\etb@noglobal\etb@err@notoggle{#1}\iffalse}%
}
\makeatother

\newtoggle{pics}
\toggletrue{pics}
%\togglefalse{pics}

\renewcommand\listingscaption{Листинг}

\usepackage{graphbox}

\hypersetup{breaklinks=true}

\let\clsCenter\Center\let\clsendCenter\endCenter
\let\Center\undefined\let\endCenter\undefined
\usepackage{ragged2e}
\let\Center\clsCenter
\let\endCenter\clsendCenter

\usepackage{numprint}
\npthousandsep{\,}

\newcommand{\CC}{C\nolinebreak\hspace{-.05em}\raisebox{.4ex}{\tiny\bf+}\nolinebreak\hspace{-.10em}\raisebox{.4ex}{\tiny\bf+}}

\usepackage{makecell}

\begin{document}
	
\input{common/renames}                 % Переопределение именований

%%% Структура диссертации (ГОСТ Р 7.0.11-2011, 4)
%\include{Dissertation/title}           % Титульный лист


\chapter*{Аннотация}

Исполнение многопоточной программы по своей природе недетерминированно, а значит стандартных методик тестирования для верификации многопоточного кода недостаточно, требуются специальные техники: fault injection и model checking.

Целью данной работы было изучить основные методы верификации конкурентного кода, выявить их недостатки и реализовать model checker, который совместил бы преимущества обоих подходов: (как fault injection) проверял бы не модель, а код на \CC, и (как model checking) перебирал бы все достижимые из теста состояния исполнения.

В работе мы опишем реализацию разработанного model checker-а и продемонстрируем его работоспособность на двух нетривиальных примерах: lock-free аллокаторе и фреймворке экзекьюторов.



\include{Dissertation/contents}        % Оглавление


\ifnumequal{\value{contnumfig}}{1}{}{\counterwithout{figure}{chapter}}
\ifnumequal{\value{contnumtab}}{1}{}{\counterwithout{table}{chapter}}


\chapter{Введение}\label{ch:ch1}

Данная работа посвящена верификации конкурентного кода.

Опыт показывает, что даже тщательно изученный конкурентный код может скрывать в себе ошибки:

“There is a rather large body of sad experience to indicate that a concurrent program can withstand very careful scrutiny without revealing its errors.” \autocite{Liveness}

Ошибки в конкурентном коде могут быть длинными, требующими определенного порядка действий, и в рабочей среде воспроизводиться крайне редко в силу своей недетерминированной природы. 

Поэтому критически важно иметь инструменты, которые будут помогать разработчику находить, понимать и устранять такие ошибки.

В этой главе мы сформулируем задачу верификации конкурентного кода, разберем преимущества и недостатки существующих подходов, и определим цели работы.

\section{Задача верификации}

\emph{Верификация программы} – это проверка соответствия программы предъявленным к ней требованиям \autocite{Intro}.

Нас будут интересовать программы, состоящие из нескольких потоков, взаимодействующих друг с другом через разделяемую память и синхронизирующие к ней доступ с помощью атомиков, мьютексов и условных переменных. Такие программы порождают \emph{конкурентные исполнения}, которые могут быть описаны в модели чередования потоков.

Еще точнее, мы хотим верифицировать не сами многопоточные программы (которые могут быть сколь угодно большими и сложными), а изолированные примитивы синхронизации, конкурентные структуры данных и базовые инфраструктурные компоненты. На практике весь нетривиальный конкурентный код (а вместе с ними и нетривиальные конкурентные баги) изолирован именно в подобных компонентах.

Примеры примитивов синхронизации: мьютексы, спинлоки, read-write локи, семафоры и барьеры.

Примеры конкурентных объектов: lock-free структуры данных (стек, очередь, хеш-таблица).

Примеры инфраструктурных компонентов:

\begin{itemize}

\item	Экзекьюторы для асинхронного исполнения задач (пулы потоков и декораторы над ними).

\item	Аллокаторы.
\end{itemize}

Требования формулируются в виде \emph{свойств}. В случае конкурентных объектов нас будут интересовать свойства, которые выполняются или нарушаются на отдельных исполнениях. Эти свойства можно разделить на два больших класса:

\begin{itemize}
\item	\emph{Safety (безопасность)} – никогда не случится ничего плохого.

\item	\emph{Liveness (живучесть)} – когда-нибудь случится нечто хорошее.
\end{itemize}

Это неформальные определения \autocite{Proving}, строгие формулировки приведены в \autocite{SafetyLiveness}.

Примеры safety-свойств:

\begin{itemize}
\item	\emph{Mutual exclusion} – два потока не могут одновременно находиться в критической секции.

\item	Память, отданная под узел lock-free стека, не может быть в то же время доступна для аллокации.
\end{itemize} 

Все эти свойства – \emph{инварианты} – утверждения про мгновенное состояние исполнения.

Примеры liveness-свойств:

\begin{itemize}

\item	\emph{Deadlock freedom} – если несколько потоков хотят захватить свободный мьютекс, то какой-нибудь из них в итоге войдет в критическую секцию.

\item	\emph{Starvation freedom} – все потоки когда-нибудь получат доступ к ресурсу.

\end{itemize}



\section{Подходы}

Детерминированных тестов для верификации конкурентных примитивов недостаточно: планировщик операционной системы недетерминированно переключает потоки, поэтому один и тот же код порождает несколько возможных вариантов исполнения.

Два основных подхода к верификации конкурентного кода – \emph{fault injection} и \emph{model checking}.


\subsection{Fault Injection}

Подход fault injection состоит в следующем: для примитива синхронизации или конкурентной структуры данных пишется \emph{стресс-тест} и в его исполнение недетерминированно внедряются \emph{сбои}: перепланирование (\mintinline{c++}{yield}), пауза (\mintinline{c++}{sleep}) или парковка потока. 

Также при контроле над средой исполнения сбои можно внедрять непосредственно в планировщик: рандомизировать run queue или очереди ожидания в примитивах синхронизации. Этот подход реализован в планировщике языка Go в режиме поиска гонок \autocite{Go}.

Fault injection – это механизм ускорения времени: в сравнении с обычным исполнением под управлением планировщика операционной системы мы перебираем больше нетривиальных сценариев за меньшее время.

Fault injection повышает вероятность нарушения инварианта в исполнении стресс-теста, но все же не может перебрать все возможные исполнения, а значит не может гарантировать корректность реализации тестируемого объекта.


\subsection{Model Checking}

Альтернативный подход – model checking – состоит в переборе всех возможных исполнений конкурентного кода. 

Для model checking-а пишется формальная \emph{спецификация}, которая состоит из

\begin{itemize}
\item	описания конкурентного объекта,

\item	теста для него и

\item	набора свойств.
\end{itemize}

Спецификация задает направленный \emph{граф состояний}, вершинами в котором являются все состояния исполнения, достижимые при исполнении теста, а дуги соответствуют атомарным операциям. Пути в этом графе называются \emph{траекториями}.

\emph{Model checker} по спецификации строит этот граф и проверяет на его состояниях или в общем случае путях (траекториях) заданные пользователем свойства. 

Если свойство нарушилось, то model checker предъявляет соответствующую этому нарушению траекторию: путь для safety-свойств и цикл для liveness-cвойств.

У fault injection-а и model checking-а схожие цели – исследовать граф состояний теста. Но, в отличие от fault injection-а, который перебирает отдельные траектории в графе состояний, model checking анализирует \emph{все} исполнения теста.

%\pagebreak

\subsection{+Cal / TLC}

Интерес индустрии к model checking-у начался после статьи “Use of Formal Methods at Amazon Web Services” \autocite{AWS}. В ней инженеры AWS рассказали про свой опыт верификации многопоточных структур данных и распределенных алгоритмов, лежащих в основе облачной инфраструктуры Amazon:
 
\begin{figure}
	\centerfloat{
		\includegraphics[scale=0.4]{aws}
	}
	\caption{Применение TLA+ / +Cal в AWS.}\label{fig:aws}
\end{figure}

Для верификации инженеры AWS выбрали языки спецификации TLA+ / +Cal и model checker TLC \autocite{Tla}. В работе мы будем говорить про +Cal, так как именно этот язык используется для спецификации многопоточного кода.

Спецификация на +Cal представляет собой псевдокод, напоминающий, в зависимости от выбранного варианта синтаксиса, код на языке C или Pascal. В качестве примера приведем фрагмент спецификации lock-free стека, написанного на C-синтаксисе языка +Cal:

\begin{figure}
\centerfloat{
\begin{tabular}{p{0.5\textwidth} p{0.5\textwidth}}
	\centering
	\vspace{0pt} 
    
    \fbox{\parbox[t][.28\textheight]{0.48\textwidth}{\includegraphics[width=0.48\textwidth]{specalloc}}}
    
	 &
	\vspace{0pt} 
    
    \fbox{\parbox[t][.28\textheight]{0.48\textwidth}{\includegraphics[width=0.35\textwidth]{spectest}}}
    
	\\
	\hfil Фрагмент алгоритма & \hfil Тест
\end{tabular}
}
\bigskip
\caption{Фрагмент спецификации lock-free стека на +Cal.}
\end{figure}



Этот псевдокод автоматически транслируется в формальный язык TLA+ и проверяется model checker-ом TLC. 

На сегодняшний день +Cal / TLC широко используется в индустрии для верификации конкурентных и распределенных алгоритмов: 

\begin{itemize}
	
\item	Спецификация конкурентных алгоритмов помогла найти баги в ядре Linux \autocite{LinuxConf} \autocite{LinuxSpec}.

\item	В Яндексе с помощью +Cal / TLC воспроизвели сценарий ABA в lock-free кеше аллокатора памяти \autocite{YaAlloc} и верифицировали исправленную версию алгоритма \autocite{YaSpec}.

\item	В Microsoft использовали +Cal для спецификации моделей согласованности распределенной базы данных CosmosDB \autocite{MsSpec}.

\item	В CockroachLabs использовали +Cal / TLC для верификации алгоритма коммита распределенных транзакций в базе данных CockroachDB \autocite{CockroachSpec}.

\end{itemize}


\subsection{Недостатки +Cal / TLC}

Хотя +Cal / TLC и являются де-факто стандартным языком спецификации / model checker-ом в индустрии, этот инструмент имеет ряд серьезных недостатков.

Самый главный и очевидный: +Cal – не язык программирования. Верифицируемый код приходится транслировать в псевдокод, а недостающие сущности – моделировать:

\begin{itemize}

\item   В +Cal нет таких сущностей как куча и указатели, их приходится заменять массивами и индексами:

\begin{figure}
	\centerfloat{
		\fbox{\includegraphics[scale=0.5]{calmem}}
	}
	\bigskip
	\caption{Моделирование памяти стека в +Cal.}\label{fig:calmem}
\end{figure}

\item   В +Сal нет атомарных операций. Атомарность определяется \emph{метками (labels)}: все действия в пределах одной метки являются атомарным шагом с точки зрения model checker-а. 

\end{itemize}

В результате такой трансляции можно упустить сценарий конкуренции, в котором нарушается инвариант.

Например, Лесли Лэмпорт, автор TLA+ и +Cal, в статье про проверку многопоточного алгоритма на +Cal признается, как случайно опустил метку и проверял фактически другой алгоритм: “... There is an amusing footnote to this story. After doing the checking, I noticed that I had inadvertently omitted a label from the pushRight operation, letting one atomic action access two shared variables.” \autocite{LamportPLuscal}

Траектории, найденные TLC, трудно читать. +Сal перед проверкой транслируется в спецификацию на TLA+, фактически – логический “ассемблер”, и TLC работает уже с этим “ассемблером”. В результате траектория состоит из низкоуровневого снимка стеков вызовов, служебных регистров (\mintinline{c++}{pc}) и всех переменных, в нем много нерелевантной информации. От исходного кода в траектории остаются только названия меток.

\begin{figure}
	\centerfloat{
		\includegraphics[scale=0.3]{caltrace}
	}
	\bigskip
	\caption{Траектория в TLA+ Toolbox.}\label{fig:caltrace}
\end{figure}

Наконец, у +Cal / TLC довольно высокий порог входа: пользователю придется изучить новый язык и разобраться с основами темпоральной логики и TLA+.

\section{Цели работы}

Для перебора всех состояний исполнения не обязательно писать на псевдокоде. Вполне можно выполнять этот перебор и над настоящим кодом. 

Цель работы – реализовать model checker, который:

\begin{itemize}
\item	{[}\emph{как +Cal / TLC}{]} Перебирает все исполнения теста.

\item	{[}\emph{как +Cal / TLC}{]} Печатает траекторию, которая приводит к нарушению инварианта. 

\item	{[}\emph{как fault injection}{]} Проверяет код на \CC, а не его модель.

\item	Показывает настоящие стектрейсы и состояния локальных переменных. 

\item	Не требует от пользователя специальных знаний в области model checking-а.
\end{itemize}

Ограничения:

\begin{itemize}
\item	Мы будем проверять не программы, а изолированные примитивы синхронизации или базовые инфраструктурные компоненты.

\item	Мы будем верифицировать только выполнение инвариантов (которые в коде тестов выражаются в виде assert-ов), и оставляем за границами работы liveness-свойства (например, livelock-и) и более сложные safety-свойства.

\item	Мы будем “разветвлять” исполнение только в точках обращения к примитивам синхронизации / атомикам. Мотивация будет приведена во второй главе работы. 

\item	Мы будем поддерживать только последовательно согласованные программы и игнорировать слабые модели памяти (\mintinline{c++}{memory_order} $\ne$ \mintinline{c++}{sec_cst}).
\end{itemize}


\section{План работы}

Во второй главе мы разберем устройство реализованного model checker-а: как перебирать исполнения многопоточного кода, как бороться с экспоненциальным ростом числа состояний, какие трудности возникают при проверке кода на \CC. 

В третьей главе мы продемонстрируем работоспособность model checker-а на двух нетривиальных примерах: lock-free аллокаторе и фреймворке экзекьюторов.

В четвертой главе мы посмотрим на инструменты, которые позволяют пользователю управлять атомарностью и отсекать ветки при переборе, выражать недетерминизм в коде теста, детализировать траекторию. 



\chapter{Model checker}\label{ch:ch2}

В этой главе мы разберем устройство реализованного \autocite{CheckerRepo} model checker-а.

\section{Управление исполнением}

Чтобы перебирать состояния исполнения, нужно:

\begin{itemize}
\item	делать снимки текущего состояния исполнения в точках обращения к примитивам синхронизации / атомикам;

\item	восстанавливать это состояние из снимка;

\item	управлять очередностью исполнения потоков (управлять планировщиком).
\end{itemize}

Для этого будем пользоваться \emph{файберами} – кооперативными потоками, реализованными в пространстве пользователя. По сути, мы продублируем в пространстве пользователя всю механику исполнения из ядра операционной системы: управление очередью исполнения и очередями ожидания в примитивах синхронизации, процедуру переключения контекста исполнения и т.п.

Файберы исполняются в одном потоке операционной системы, что исключает влияние на моделируемое исполнение недетерминизма системного планировщика.

Файбер хранит в себе:

\begin{itemize}

\item	Стек – заранее аллоцированный регион памяти.

\item	Контекст – callee-saved регистры, instruction pointer, stack pointer (имеет смысл только для остановленного файбера).

\item	Enum состояния: файбер готов исполняться (\mintinline{c++}{runnable}) / заснул в очереди ядра (\mintinline{c++}{suspended}) / завершил исполнение (\mintinline{c++}{terminated}).

\item	Пользовательскую функцию.

\end{itemize}

Переключение контекста файбера реализовано на ассемблере и состоит из следующих шагов:

\begin{enumerate}

\item	На стек файбера сохраняется контекст исполнения: значения callee-saved регистров.

\item	\mintinline{c++}{rsp} текущего файбера фиксируется в поле файбера для восстановления контекста в будущем.

\item	\mintinline{c++}{rsp} переключается на стек нового файбера.

\item	С нового стека восстанавливается контекст (значения callee-saved регистров).

\end{enumerate}

Таким образом, в файбере в поле с контекстом фактически хранится лишь указатель \mintinline{c++}{rsp}, остальные регистры хранятся на стеке.

\textbf{Замечание:} в работе мы будем использовать термин \emph{файберы}, когда речь идет о реализации model checker-а, и \emph{потоки}, когда речь идет о наблюдаемом поведении теста / свойствах конкурентных исполнений. Пользователь model checker-а работает только с потоками (через объект \mintinline{text}{thread}) и никаких файберов не наблюдает.


\section{Снимки состояния}

Состояние исполнения образовано:

\begin{itemize}
\item	состоянием каждого файбера (стек, регистры, enum состояния, исполняемая функция);

\item	состоянием динамической памяти (кучи).
\end{itemize}

Снимок состояния делается в контексте model checker-а, т.е. в момент, когда ни один из файберов не исполняется.

Снимок регистров отдельно делать не нужно: при переключении контекста нужные регистры сохраняются на вершину стека остановленного файбера, так что достаточно сохранять только сами стеки.

Сохраняется только использованная часть стека – до адреса \mintinline{c++}{rsp} (он доступен через сохраненный контекст исполнения).

Между стеками файберов / стеком и кучей могут быть ссылки, про которые model checker ничего не знает (один файбер захватывает мьютекс, который расположен на стеке другого файбера / файбер создает \mintinline{c++}{shared_ptr} на объект), поэтому память для стеков и кучи выделяется в model checker-е один раз, и при каждом восстановлении снимка стеки и куча восстанавливаются по своим прежним адресам. 

Все аллокации в тесте перехватываются, и если аллокация выполняется непосредственно пользовательским кодом (а не model checker-ом), то ее обрабатывает специальный аллокатор. Аллокатор поддерживает свободные блоки в интрузивном односвязном списке внутри заранее выделенной арены памяти. Снимок аллокатора – это использованный диапазон арены и два указателя: на голову списка свободных блоков и на границу использованного префикса арены.

Служебные аллокации, выполняемые в контексте файбера (например, при запуске нового потока-файбера), отделяются от пользовательских аллокаций с помощью расставленных в библиотеке \mintinline{c++}{AllocationGuard}-ов. \mintinline{c++}{AllocationGuard} выключает аллокатор model checker-а в области своей жизни.

Итого, снимок состояния в model checker-е образован:

\begin{itemize}
\item	стеком, enum-ом состояния и исполняемой функцией каждого файбера;

\item	снимком кучи.
\end{itemize}

\begin{figure}
	\centerfloat{
		
		
					\begin{tabular}{c c c}
						%\centering 
						\fbox{{\includegraphics[align=c,width=0.35\textwidth]{snap1}}}
						&
						${\Longrightarrow}$
						& \fbox{{\includegraphics[align=c,width=0.35\textwidth]{snap2}}}
						\\
						\hfil Память (куча и стеки) & & \hfil Снимок
					\end{tabular}
			
	}
	\bigskip
	\caption{Снятие снимка состояния.}
\end{figure}

\section{Перебор исполнений}

Чтобы находить кратчайшие нарушения инвариантов, будем обходить граф состояний исполнения поиском в ширину. Граф не материализуется в памяти явно, вместо этого мы умеем из каждого состояния получать все смежные с ним.

\subsection{Обход графа состояний}

Посмотрим, как model checker перебирает исполнения теста.

Ключевой объект model checker-а – это очередь состояний, продолжения которых еще нужно посетить, фронт обхода в ширину. 

На старте model checker создает первый файбер, который будет исполнять код теста, и кладет снимок начального состояния в очередь.

Теперь посмотрим на цикл обхода графа состояний. На очередной итерации model checker достает из очереди снимок состояния, перебирает все файберы, которые в этом состоянии могут продолжить работу, и для каждого из них исполняет один \emph{шаг} (\mintinline{c++}{Step}). 

\iftoggleverb{pics}

\begin{figure}
	\centerfloat{
		\fbox{\includegraphics[scale=0.5]{runloop}}
	}
	\bigskip
	\caption{Цикл model checker-а.}\label{fig:runloop}
\end{figure}

\else

\begin{listing}
	\centering
	
	\begin{minted}{c++}
void Checker::RunLoop() {
	while (!states_.Empty()) {
		State state = states_.Pop();

		const size_t fiber_count = fibers_.size();
		for (size_t i = 0; i < fiber_count; ++i) {
			if (!state.Enabled(i)) {
				continue;
			}
			Step(state, i);
		}
	}
}
	\end{minted}
	\caption{Цикл model checker-а.}
	\label{loop}
\end{listing}

\fi

Перебор \mintinline{c++}{runnable} файберов для каждого снимка – это и есть ветвление исполнения, по сути мы рандомизируем run queue в планировщике.

\subsection{Шаг исполнения}

Шаг исполнения файбера заключен между точками обращения к разделяемым объектам (примитивам синхронизации / атомикам / аллокатору памяти) и реализуется парой служебных функций \mintinline{c++}{Step} / \mintinline{c++}{Fork}.

Начинается шаг с вызова функции \mintinline{c++}{Step}: в контексте model checker-а из снимка восстанавливается состояние исполнения (куча и стеки накладываются на соответствующие регионы памяти), затем контекст переключается на выбранный файбер (который ранее остановился в вызове \mintinline{c++}{Fork}).

Далее файбер исполняется до тех пор, пока не встретит следующий вызов функции \mintinline{c++}{Fork}.

\iftoggleverb{pics}

\begin{figure}
	\centerfloat{
		\fbox{\includegraphics[scale=0.4]{step}}
	}
	\bigskip
	\caption{Шаг файбера (\mintinline{c++}{Step}).}\label{fig:step}
\end{figure}

\else

\begin{listing}
	\centering
	
	\begin{minted}{c++}
    
void Checker::Step(State& prev, size_t index) {
  Restore(prev);
  SwitchTo(fibers_[index]);
  // fiber running until next Fork...
  UpdateStates(prev, index);
}

	\end{minted}
	\caption{Шаг файбера (\mintinline{c++}{Step}).}
	\label{step}
\end{listing}

\fi

Вызов \mintinline{c++}{Fork} – точка ветвления, в ней может произойти переключение на произвольный \mintinline{c++}{runnable} файбер. Вызов \mintinline{c++}{Fork} переключает контекст файбера на контекст model checker-а, тот делает снимок состояния исполнения и помещает его в очередь поиска в ширину.

\begin{figure}
	\centerfloat{
		\includegraphics[scale=0.5]{forkblack}
	}
	\caption{Развилка в коде (вызов \mintinline{c++}{Fork}).}\label{fig:fork}
\end{figure}

\subsection{Точки ветвления}

Будем предполагать, что реализация верифицируемого конкурентного объекта не содержит data race-ов, т.е. проверяемый тест – программа, \emph{свободная от гонок (data-race-free)}. 

Если программа свободна от гонок, то потоки в ней не могут наблюдать относительный порядок действий между точками синхронизации в другом потоке (т.е. между обращениями к атомикам, мьютексам и другим объектам, через которые реализуется отношение synchronizes-with).

Это наблюдение упрощает перебор model checker-а: ветвления исполнения (вызовы \mintinline{c++}{Fork}) можно встраивать только при обращении к объектам, через которые осуществляется синхронизация.

Чтобы перехватывать обращения к атомикам, мьютексам и условным переменным и ветвить в них исполнения, мы пишем собственные версии этих примитивов и требуем от пользователя model checker-а, чтобы он использовал их вместо стандартных (для этого в проверяемом коде придется заменить пространство имен с \mintinline{c++}{std} на пространство имен библиотеки model checker-а, на данный момент это \mintinline{c++}{twist::stdlike}).

\iftoggleverb{pics}

\begin{figure}
	\centerfloat{
		\fbox{\includegraphics[scale=0.5]{load}}
	}
	\bigskip
	\caption{Собственная реализация \mintinline{c++}{load} у атомика.}\label{fig:lock}
\end{figure}

\else

\begin{listing}
	\centering
	
	\begin{minted}{c++}
    
void Mutex::Lock() {
  Fork();
  while (locked_) {
    wait_queue_.Park();
  }
  locked_ = true;
}

	\end{minted}
	\caption{Реализация метода lock у мьютекса.}
	\label{lock}
\end{listing}

\fi

Аналогично, для перехвата динамических аллокаций мы переопределяем операторы \mintinline{text}{new} / \mintinline{text}{delete}: если находимся в контексте пользовательского кода, вызываем аллокатор model checker-а, иначе – \mintinline{c++}{malloc} / \mintinline{c++}{free}.

\subsection{Run queue / Wait queue}

Model checker – не планировщик, в нем нет явной очереди \mintinline{c++}{runnable} файберов (run queue), для каждого снимка запускается каждый \mintinline{c++}{runnable} файбер. 

Аналогично мы поступаем с блокирующими примитивами синхронизации, которые внутри имеют очереди ожидания (мьютексы, условные переменные): 

Явно такие очереди не моделируются, вместо этого в каждом файбере хранится адрес очереди, в которой он заснул. Когда, например, мьютекс в \mintinline{c++}{unlock} будит один из заблокированных на нем файберов, model checker создает сразу несколько веток исполнения – по ветке на каждый файбер, у которого в поле адреса хранится текущая очередь.

В планировщике Go \autocite{Go} при поиске гонок происходит рандомизация очередей ожидания и исполнения. У нас же случайность трансформируется в перебор всех возможных исполнений.


\subsection{Очередь во внешней памяти}

С ростом исследуемой глубины исполнения количество состояний будет расти экспоненциально, фронт поиска в ширину быстро станет огромным (в одном из тестов размер фронта достигал десятков миллионов состояний) и перестанет помещаться в оперативную память.

Поэтому очередь для поиска в ширину в model checker-е реализована во внешней памяти. Вот как она устроена:

Все состояния в очереди делятся на две непрерывные группы (фронта):

\begin{enumerate}

\item	\emph{Старый фронт} – состояния наименьшей глубины ($d$). 

\item	\emph{Новый фронт} – состояния, полученные из старого фронта (глубины $d+1$). 

\end{enumerate}
	
Пока не исчерпался старый фронт, в очереди не могут появиться состояния глубины $d+2$.
 
Будем поддерживать по файлу на каждый из двух фронтов: один для чтения старого фронта, другой – для записи нового. При смене глубины старый фронт отбрасывается, а новый – переходит в старый. 

Model checker не работает с файлами напрямую, вместо этого он использует специальный аллокатор, который работает с аренами памяти, отображенными в файлы с помощью \mintinline{c++}{mmap}-а.

Максимальный перерасход внешней памяти (в два раза) происходит при смене глубины. Но почти всегда издержки компенсируются (если следующий фронт произведет хотя бы в два раза больше состояний).


\subsection{Повторные состояния}

Чтобы не посещать одни и те же состояния повторно, model checker хеширует снимки состояний и не помещает снимок в очередь, если его хеш уже встречался ранее.

Хеширование неизбежно ведет к потере информации, поэтому model checker может упустить состояния, если произойдет \emph{коллизия} – ситуация, когда двум состояниям ставится в соответствие один хеш. 

TLC аналогичным образом использует хеширование состояний. Кроме того, он оценивает вероятность коллизии и дает возможность пользователю перезапустить перебор с другой хеш-функцией:

“TLC saves only 64-bit fingerprints (hashes) of the reachable states that it finds, not the complete states.  If two different reachable states have the same fingerprint, a situation called a collision, TLC may not find all reachable states.  At the end of a run, TLC prints estimates of how likely it was that a collision occurred.” \autocite{TlcOptions}

\begin{figure}
	\centerfloat{
		\fbox{\includegraphics[scale=0.5]{fingerprint}}
	}
	\bigskip
	\caption{Вероятность коллизии, показанная в TLA+ Toolbox.}\label{fig:fingerprint}
\end{figure}

Но вероятность коллизии невелика, и на практике хеширование не мешает model checker-у находить баги, зато сильно ускоряет проверку. 

Для хеширования model checker использует FNV-функцию \autocite{Fnv}: она быстрая и обобщается на комбинирование хешей (что удобно для хеширования состояний исполнения).
  
При хешировании участков памяти (стеки, куча) помогает развертка цикла \autocite{FasterHash}. Развертка разрывает зависимость следующей итерации цикла от предыдущей, и процессор (фактически, конвейер для инструкций) выполнит такие итерации параллельно.

\section{Симметрия}

Любой model checker сталкивается с проблемой \emph{state space explosion}. Даже если разветвляться только перед обращениями к разделяемым объектам, количество состояний будет расти экспоненциально с ростом количества атомиков / потоков. 

Но если разные потоки исполняют один и тот же код (что ожидаемо для теста мьютекса или конкурентной структуры данных), то некоторые состояния могут быть \emph{симметричны} – сводиться одно к другому с помощью перенумерации потоков. 

Чтобы учесть симметрию, model checker \emph{канонизирует} состояние: стирает со стеков зависимости от адреса конкретного потока (например, ссылки на свой же стек заменяет относительными отступами). Получившиеся стеки хешируются по отдельности, сортируются и комбинируются в один результирующий хеш.

TLC тоже предоставляет такую оптимизацию и обобщает это наблюдение до симметрии множеств \autocite{Symmetry}.

\section{Проверка инвариантов}

Цель model checker-а – найти нарушение инварианта в некотором исполнении теста или убедиться в отсутствии таких нарушений во всех исполнениях.

Есть две вариации инвариантов:

\begin{itemize}

\item	Локальный инвариант – аналог обычного assert-а, задается в коде теста макросом \mintinline{c++}{CHECKER_ASSERT(condition)}. Проверка выполняется в контексте пользовательского потока. Если \mintinline{c++}{condition} не выполнен, то управление передается model checker-у, и тот сообщает об ошибке. 

\item	Глобальный инвариант – произвольный предикат на разделяемом состоянии. Инвариант проверяется в контексте model checker-а: проверка запускается на каждое новое состояние в исполнении и происходит атомарно (с отключенными \mintinline{c++}{Fork}-ами). Глобальный инвариант задается в начале теста с помощью \mintinline{c++}{AddInvariant(predicate)}.

\end{itemize}

Для проверки взаимного исключения в мьютексе удобнее поставить \mintinline{c++}{CHECKER_ASSERT} в критической секции: 

\iftoggleverb{pics}

\begin{figure}
	\centerfloat{
		\fbox{\includegraphics[scale=0.5]{checkerassert}}
	}
	\bigskip
	\caption{Локальный инвариант.}
\end{figure}

\else

\begin{listing}
	\centering
	
	\begin{minted}{c++}
    
while (true) {
  if (lock.TryLock()) {
    CHECKER_ASSERT(!in_critical_section.exchange(true), "Mutual exclusion violated");
    in_critical_section.store(false);
    lock.Unlock();
  }
}

	\end{minted}
	\caption{Локальный инвариант.}
	\label{local}
\end{listing}

\fi

Но инварианты lock-free стека уже трудно проверить локально, для него удобнее использовать глобальный вариант (например, он может проверять, что каждый узел стека действительно аллоцирован (см. \ref{app:A3})):

\iftoggleverb{pics}

\begin{figure}
	\centerfloat{
		\fbox{\includegraphics[scale=0.7]{globalassert}}
	}
	\bigskip
	\caption{Глобальный инвариант.}
\end{figure}

\else

\begin{listing}
	\centering
	
	\begin{minted}{c++}

TEST_SUITE(AllocLockFreeStack) {
  SIMPLE_CHECKER_TEST(Spec) {
    MemoryPool pool{3};
    LFAllocator stack{pool};
        
    AddInvariant([&stack]() -> std::pair<bool, std::string> {
      // check every node of stack...
      // validation failed -> return {false, report}
    });
	
    // test code...
  }
}
	\end{minted}
	\caption{Глобальный инвариант.}
	\label{localinv}
\end{listing}

\fi

В model checker-е реализован глобальный служебный инвариант – проверка \emph{взаимной блокировки (deadlock)}. Взаимная блокировка с точки зрения model checker-а – достижимое состояние, в котором еще есть живые потоки, но нет ни одного потока, готового исполняться (т.е. в состоянии \mintinline{c++}{runnable}). Как и TLC, model checker проверяет этот инвариант по умолчанию.  


\section{Печать траектории}

Если при проверке теста model checker находит состояние, в котором нарушается пользовательский или служебный инвариант, то он завершает перебор и печатает найденную траекторию. Пример найденной траектории приведен в приложении \ref{app:trace}.

Проверка инварианта и печать траектории разделены между двумя режимами сборки теста: \mintinline{c++}{Check} и \mintinline{c++}{Trace}.

В режиме \mintinline{c++}{Check} model checker обходит граф состояний исполнения и с каждым снимком состояния хранит траекторию, которая привела к этому состоянию. Траектория детерминирует выбор очередного потока в каждой развилке исполнения и кодируется последовательностью индексов потоков: на $i$-ой позиции записан номер потока, который был выбран на $i$-ом вызове \mintinline{c++}{Fork}. В случае нарушения инварианта найденная траектория сохраняется на диск.

В режиме \mintinline{c++}{Trace} model checker считывает с диска найденную в первом режиме траекторию и проигрывает ее, показывая для каждой развилки разделяемое состояние, стеки потоков и локальные переменные.  

Траектория разделена на секции (\emph{runs}), каждая секция – непрерывный фрагмент исполнения одного потока, который ограничен переключением контекста / стартом потока (слева) / завершением потока (справа).

Каждая секция показывает состояние потока в нескольких временных точках: снимки состояния до и после исполнения секции, и промежуточные снимки в точках обращения к атомикам / примитивам синхронизации / аллокатору памяти.

Для каждого шага в секции печатается:

\begin{itemize}
\item	Разделяемое состояние. Model checker не знает, как устроено и как печатать разделяемое состояние, пользователь пишет процедуру печати самостоятельно и через \mintinline{c++}{PrintState} в начале теста передает ее model checker-у.

\begin{figure}
	\centerfloat{
		\fbox{\includegraphics[scale=0.5]{sharedstate}}
	}
	\bigskip
	\caption{Разделяемое состояние.}
\end{figure}

\item	Стектрейсы потока. Стектрейсы собираются с помощью библиотеки \mintinline{c++}{backward-cpp} \autocite{Backward}. Служебные фрагменты стектрейса (вызовы внутри model checker-а), код стандартной и сторонних библиотек фильтруется.

\item	Локальные переменные. В точке, где мы хотим снять значения локальных переменных, мы вызываем \mintinline{c++}{fork} и из дочернего процесса присоединяемся отладчиком (\mintinline{c++}{gdb}) к родительскому. В планах – использовать программный интерфейс \mintinline{c++}{lldb} вместо запуска дебаггера.

\begin{figure}
	\centerfloat{
		\fbox{\includegraphics[scale=0.5]{stacklocals}}
	}
	\bigskip
	\caption{Стектрейсы и локальные переменные.}
\end{figure}

\item	Заметки о служебных событиях во время исполнения потока. Добавляются с помощью макроса \mintinline{c++}{SHOW_NOTE}.

\begin{figure}
	\centerfloat{
		\fbox{\includegraphics[scale=0.5]{running}}
	}
	\bigskip
	\caption{Заметки о событиях.}
\end{figure}

\end{itemize}

\section{Инженерные препятствия}

Model checker проверяет код, оптимизируемый компилятором. Поэтому нужно решать ряд трудностей, связанных непосредственно с языком программирования / оптимизациями компилятора.

\subsection{Абсолютные адреса и симметрия}

Для учета симметрии model checker канонизирует стеки, но этому процессу мешают абсолютные адреса:

\begin{itemize}
\item	При вызове функции на стек сохраняется указатель на начало родительского фрейма (значение регистра \mintinline{c++}{rbp}).

\item	На стеке может лежать ссылка на переменную, аллоцированную на этом стеке (в виде другой переменной или в виде значения регистра).

\item	В вызове \mintinline{c++}{Fork} перед переключением на контекст model checker-а вызывается служебная функция \mintinline{c++}{GetCurrentFiber}. В результате на стеке может “отпечататься” адрес текущего файбера.

\item	Файбер хранит исполняемую лямбду в type-erased контейнере. Для механизма type erasure требуется аллокация на куче. Так как такая аллокация происходит независимо для каждого файбера, все контейнеры получат свой собственный уникальный адрес в памяти, даже если отвечают одной лямбде. 
\end{itemize}

Канонизация очищает стек от внутренней адресации, адреса текущего файбера и адреса контейнера для исполняемой пользовательской функции.

Это только эвристика: model checker не может отличить адрес от числа, попавшего в тот же диапазон. Но на практике такие числа редко используются программой, и model checker находит баги.

\subsection{Встраивание вызовов}

Еще одна проблема – учет одинаковых состояний в бесконечных циклах. Бесконечные циклы возникают естественным образом: например, в тесте мьютекса поток в таком цикле захватывает и отпускает мьютекс.

Ожидается, что на каждой итерации стек файбера будет отличаться только состоянием локальных переменных, объявленных перед циклом. Поэтому граф состояний исполнения должен быть небольшим. 

Рассмотрим проблему на вышеупомянутом примере (захват и освобождение мьютекса в бесконечном цикле). Компилятор может встроить вызов \mintinline{c++}{lock} в итерацию цикла. Тогда инструкции \mintinline{c++}{lock}-а “засорят” регистры в фрейме цикла. В начале следующей итерации, когда вызовется \mintinline{c++}{Fork}, “мусорные” значения callee-saved регистров сохранятся на стеке – model checker сделает снимок формально другого состояния.

Но по сути поток вернулся к тому же состоянию, с которого начинал. Если бы вызов не встроился в цикл, то метод \mintinline{c++}{lock} на выходе восстановил бы значения тех callee-saved регистров, которыми пользовался. Значит, на следующей итерации при вызове \mintinline{c++}{Fork} на стек сохранились бы те же значения регистров, что и на предыдущей. Новое состояние бы не создалось.

Есть два решения проблемы:

\begin{itemize}
\item	Вместо цикла указать в конце итерации \mintinline{c++}{RestartFiber()} – служебную функцию model checker-а, которая полностью очищает стек и перезапускает исполняемую файбером функцию. Подходит только если между итерациями на стеке не сохраняется промежуточных переменных. 

\item	Запретить компилятору “засорять” фрейм цикла: поставить вызываемым из цикла методам атрибут \mintinline{c++}{noinline}.
\end{itemize}

\subsection{Выравнивание стека}

По ABI стек должен быть выровнен перед исполнением инструкции \mintinline{asm}{call}. Чтобы это сделать, компилятору выгоднее не двигать \mintinline{c++}{rsp} напрямую, а подвинуть \mintinline{c++}{rsp} с помощью \mintinline{asm}{push %rax} \autocite{Align}. Такой \mintinline{asm}{push} совершается с регистром, в котором хранится возвращаемое значение. Поэтому, даже если цикл использует только \mintinline{c++}{noinline}-функции, вызовы таких функций могли оставить на стеке следы результата предыдущей итерации. 

Решение – заставить компилятор выравнивать стек консервативно через атрибут \mintinline{c++}{force_align_arg_pointer}. Он служит для других целей, но подошел и для нашей – компилятор для функций с таким атрибутом будет изменять \mintinline{c++}{rsp} напрямую.


\chapter{Тестирование}\label{ch:ch3}

Реализованный model checker способен искать настоящие, нетривиальные ошибки в конкурентном коде. В этой главе мы продемонстрируем это утверждение на двух примерах: 

\section{LFAlloc}

Первый пример – это lock-free стек, выполняющий роль глобального кеша блоков в аллокаторе LFAlloc, который используется в алгоритме машинного обучения CatBoost и внутренних системах Яндекса \autocite{YaAlloc}. Реализация стека в ранней версии аллокатора содержала сложный сценарий ABA из нескольких десятков шагов. Уже после обнаружения ошибки для стека написали спецификацию на +Cal \autocite{YaSpec}, с помощью которой подтвердили наличие ошибки и верифицировали патч для нее.

Наш model checker находит тот же баг, но работает непосредственно с реализацией конкурентной структуры данных, а значит, не надо специфицировать стек на псевдокоде, расставлять метки атомарности, моделировать память и указатели.

%\iftoggleverb{pics}

\begin{figure}
	\centerfloat{
		\begin{tabular}{p{0.5\textwidth} p{0.5\textwidth}}
			\centering
			\vspace{0pt} \fbox{\parbox[t][.28\textheight]{0.48\textwidth}{\includegraphics[width=0.48\textwidth]{specalloc}}}
			&
			\vspace{0pt} \fbox{\parbox[t][.28\textheight]{0.48\textwidth}{\includegraphics[width=0.48\textwidth]{checkeralloc}}}
			\\
			\hfil +Cal & \hfil \CC
		\end{tabular}
	}
	\bigskip
	\caption{Сравнение спецификации и кода на примере lock-free стека.}
\end{figure}

%\else
\begin{comment}

\begin{figure}
	\centerfloat{
		\begin{tabular}{p{0.5\textwidth} p{0.5\textwidth}}
			\centering
			
			\vspace{0pt}
			
			\fbox{\parbox[t][.28\textheight]{0.48\textwidth}{\includegraphics[width=0.48\textwidth]{specalloc}}}
			&
			\vspace{0pt} 
			
			\begin{minted}[fontsize=\tiny, frame=single, linenos=false]{c++}
 bool Alloc(Node*& node) {
	Node* pending_head = pending_list_.Top();
	if (allocation_count_.fetch_add(1) == 0) {
		if (pending_head != nullptr
		&& pending_list_.TryClear(pending_head)) {
			node = pending_head;
			pending_head = pending_head->next_.load();
			item_list_.Append(pending_head);
			allocation_count_.fetch_sub(1);
			return true;
		}
	}
	node = item_list_.Pop();
	allocation_count_.fetch_sub(1);
	return node != nullptr;
}
			\end{minted}
			
			
			\\
			\hfil +Cal & \hfil \CC
		\end{tabular}
	}
	\bigskip
	\caption{Сравнение спецификации и кода на примере lock-free стека.}
\end{figure}

%\fi
\end{comment}

При этом структуру теста можно не менять, а позаимствовать ее из спецификации на +Cal: поток в бесконечном цикле аллоцирует и деаллоцирует блоки памяти (узлы стека).

\begin{figure}
	\centerfloat{
		\begin{tabular}{p{0.5\textwidth} p{0.5\textwidth}}
			\centering
			\vspace{0pt} \fbox{\parbox[t][.2\textheight]{0.48\textwidth}{\includegraphics[width=0.48\textwidth]{spectest}}}
			&
			\vspace{0pt} \fbox{\parbox[t][.2\textheight]{0.48\textwidth}{\includegraphics[width=0.38\textwidth]{checkertest}}}
			\\
			\hfil +Cal & \hfil \CC
		\end{tabular}
	}
	\bigskip
	\caption{Один и тот же тест для TLC и для model checker-а.}
\end{figure}

\begin{comment}

\begin{figure}
	\centerfloat{
		\begin{tabular}{p{0.5\textwidth} p{0.5\textwidth}}
			\centering
			\vspace{0pt} \includegraphics[width=0.5\textwidth]{spectest}
			&
			\vspace{0pt} 
			
			\begin{minted}{c++}
auto worker = [&stack]() {
	while (true) {
		Node* node{nullptr};

		if (stack.Alloc(node)) {
			stack.Free(node);
		}
	}
};
			\end{minted}
			
			\\
			\hfil +Cal & \hfil \CC
		\end{tabular}
	}
	\bigskip
	\caption{Один и тот же тест для TLC и для model checker-а.}
\end{figure}

\end{comment}

Реализация lock-free стека и тест для model checker-а приведены в приложении \ref{app:A}.

Траектория, напечатанная нашим model checker-ом (см. \ref{app:trace}), более читаема, чем траектория, которую печатает TLC: в ней для каждого шага печатается настоящий стектрейс и значения локальных переменных. По ней гораздо проще восстановить сценарий ABA.

Приведем статистику поиска данного бага model checker-ом и TLC:

%\begin{center}
\begin{table}
\centering
\begin{tabular}{| l | c | c |}
\hline
                   & model checker & TLC \\
                   \hline
 Длина траектории & 61 шаг & 76 шагов \\ 
 Найдено состояний & \numprint{17906188} & \numprint{26168885} \\  
 Найдено уникальных состояний & \numprint{4751588} & \numprint{7269203} \\
 Время                       & 29 cекунд & 8 минут\\
 	\hline
\end{tabular}

\bigskip
\captionsetup{justification=centering}
\caption{Статистика поиска ABA в LFAlloc.} 
\end{table}
%\end{center}

В четвертой главе будет описан инструмент model checker-а (недетерминизм, выражаемый в коде), который позволит сделать тест для стека более гибким – и, как следствие, сократить перебор и получить более короткую траекторию.

LFAlloc - пример со сложным (длинным) сценарием ABA. Следующий пример - про сложность другого рода: про поиск ошибки в большом коде.

\section{Executors}

Executors – фреймворк для асинхронного исполнения задач. Пользователю предоставляется сущность \emph{экзекьютора} (представленная интерфейсом \mintinline{c++}{IExecutor}) с единственной операцией – запланировать \emph{задачу} (произвольную функцию без аргументов и возвращаемого значения) для исполнения в некотором потоке.

Базовый экзекьютор – пул потоков фиксированного размера, разбирающих разделяемую очередь задач. Поверх этого пула можно писать декораторы, например, для последовательного исполнения задач или для поддержки приоритетов задач.

Код в таком фреймворке не похож на изолированную структуру данных: он гораздо больше и в нем много нетривильных конкурентных “деталей”: блокирующие и лок-фри очереди, спинлоки, вспомогательные примитивы синхронизации и т.п.

\iftoggleverb{pics}

\begin{figure}
	\centerfloat{
		\fbox{\includegraphics[scale=0.5]{strandfields}}
	}
	\bigskip
	\caption{Поля одного из экзекьюторов (Strand-а) – композиция конкурентных объектов.}\label{fig:strandfields}
\end{figure}

\else

\begin{listing}
	\centering
	
	\begin{minted}{c++}
    
class Strand : public IExecutor {
// implementation...

private:
	IExecutorPtr executor_;
	MPSCLockFreeQueue<Task> tasks_;
	Spinlock lock_;
};

	\end{minted}
	\caption{Поля одного из экзекьюторов (Strand-а) – композиция конкурентных объектов.}
	\label{strandfields}
\end{listing}

\fi

Рассмотрим \emph{Strand} (или \emph{Serial Executor}) – декоратор над пулом потоков, который исполняет все запланированные через него задачи в потоках декорируемого пула строго последовательно и в порядке их добавления.

В реализации этого декоратора легко допустить race condition при планировании / перепланировании служебной задачи, которая запускает пользовательские в потоках нижележащего пула. Такой race condition приведет к тому, что последняя пачка задач Strand-а может не выполниться.

\begin{figure}
	\centerfloat{
		\begin{tabular}{p{0.5\textwidth}p{0.5\textwidth}}
			\centering
			\vspace{0pt} \fbox{\parbox[t][.29\textheight]{0.48\textwidth}{\includegraphics[width=0.48\textwidth]{exec}}}
			&
			\vspace{0pt} \fbox{\parbox[t][.29\textheight]{0.48\textwidth}{\includegraphics[width=0.48\textwidth]{sched}}}
		\end{tabular}
	}
	\bigskip
	\caption{Фрагменты неправильной реализации Strand-а.}
\end{figure}

Тест на такую ошибку в model checker-е устроен просто: в Strand последовательно добавляются две задачи, пул останавливается, после чего проверяется счетчик выполненных задач:

\iftoggleverb{pics}

\begin{figure}
	\centerfloat{
		\fbox{\includegraphics[scale=0.5]{strandtest}}
	}
	\bigskip
	\caption{Тест для Strand-а.}\label{fig:strandtest}
\end{figure}

\else

\begin{listing}
	\centering
	
	\begin{minted}{c++}
    
auto tp = MakeStaticThreadPool(1);
auto strand = MakeStrand(tp);

size_t tasks = 0;
strand->Execute([&]() { ++tasks; });
strand->Execute([&]() { ++tasks; });

tp->Join();

CHECKER_ASSERT(tasks == 2, "Expected 2 completed tasks, found " << tasks);

	\end{minted}
	\caption{Тест Strand-а.}
	\label{strandtest}
\end{listing}

\fi

Несмотря на простоту теста, за счет перебора всех исполнений model checker находит описанный выше race condition.

\chapter{Инструменты пользователя}\label{ch:ch4}

В этой главе мы опишем инструменты, с помощью которых пользователь может настраивать поведение model checker-а.

\section{Атомарность}

Если верифицируемый код устроен достаточно сложно (использует сразу несколько конкурентных примитивов / структур данных), то найденная model checker-ом траектория получится длинной и потому плохо читаемой – в ней будет слишком много действий, не относящихся непосредственно к найденному race condition-у.

Например, Strand / Serial Executor использует lock-free очередь, которая генерирует в траектории много промежуточных состояний, но ошибка в Strand-е, на которой в предыдущей главе проверялся model checker, кроется не в очереди.

Другая ситуация: мы уверены в корректности той же lock-free очереди (пусть мы верифицировали ее независимо) и хотим сократить перебор в model checker-е.

В описанных ситуациях удобно предположить, что lock-free очередь линеаризуема (атомарна) и не ветвить исполнение внутри реализации ее методов.

Аннотировать линеаризуемый объект можно с помощью декоратора \mintinline{c++}{ForkGuarded<T>}, который проксирует все вызовы методов объекта:

\begin{enumerate}
\item	Вызывает \mintinline{c++}{Fork} (вызов метода – ровно одна развилка).

\item	Выключает \mintinline{c++}{Fork} на время жизни проксирующего вызов объекта.

\item	Вызывает метод объекта, внутри которого ветвлений уже не будет.
\end{enumerate}

По сути, декоратор \mintinline{c++}{ForkGuarded<T>} – это прямой аналог метки в +Cal, он позволяет регулировать гранулярность атомарности для model checker-а. Но, в отличие от меток, этот механизм более безопасный: атомарность устанавливается на объекте, а не на отдельных фрагментах кода.

\iftoggleverb{pics}

\begin{figure}
	\centerfloat{
		\fbox{\includegraphics[scale=0.5]{forkguarded}}
	}
	\bigskip
	\caption{\mintinline{c++}{ForkGuarded<T>} в Strand-е.}
\end{figure}

\else

\begin{listing}
	\centering
	
	\begin{minted}{c++}
class Strand : public IExecutor {
// implementation...

private:
	IExecutorPtr executor_;
	ForkGuarded<MPSCLockFreeQueue<Task>> tasks_;
	ForkGuarded<Spinlock> lock_;
};
	\end{minted}
	\caption{\mintinline{c++}{ForkGuarded<T>} в Strand-е.}
	
\end{listing}

\fi


\section{Either / Random}

Чтобы перебрать больше сценариев в тесте, разветвлять можно и сам код. 

Для примера возьмем спинлок с методами \mintinline{c++}{Lock} и \mintinline{c++}{TryLock}. Естественный тест для него – это захват и освобождение лока в бесконечном цикле. У потока есть два варианта, как взять лок, и хочется проверить оба. В +Cal для этого есть конструкция \mintinline{c++}{either} / \mintinline{c++}{or}: она создает развилку в коде, в которой model checker пройдет по обеим веткам исполнения:

\begin{figure}
	\centerfloat{
		\fbox{\includegraphics[scale=0.4]{caleither}}
	}
	\bigskip
	\caption{\mintinline{c++}{either} из +Cal.}
\end{figure}


В model checker-е для аналогичных целей реализована функция \mintinline{text}{bool Either()}. Вызов функции переключается на контекст model checker-а, и тот кладет в очередь состояний не один снимок, а два – каждый со своим выходом из \mintinline{c++}{Either}. Чтобы не создавать два почти одинаковых снимка состояния, model checker продолжает оба состояния еще на один шаг вперед. 

\iftoggleverb{pics}

%\begin{comment}
\begin{figure}
	\centerfloat{
		\fbox{\includegraphics[scale=0.35]{either}}
	}
	\bigskip
	\caption{\mintinline{text}{Either} model checker-а.}
\end{figure}
%\end{comment}

\else

\begin{listing}
	\centering
	
	\begin{minted}{c++}
if (Either()) {
	lock.Lock();
} else {
	lock.TryLock();
}
	\end{minted}
	\caption{Тест для спинлока с Either.}
	
\end{listing}

\fi

Можно создавать не две, а произвольное число новых веток. Такая логика реализуется функцией \mintinline{text}{size_t Random(size_t n)}: она порождает $n$ новых продолжений, каждое с уникальным возвращаемым значением от $0$ до $n-1$. 

\mintinline{c++}{Random} позволяет писать тесты с более разнообразными сценариями исполнения. В +Cal-версии теста для LFAlloc поток всегда сначала аллоцирует узел, а потом возвращает обратно. Вместо этого поток может узнать, сколько узлов уже аллоцировано ($n$), и сгенерировать “случайное” значение с помощью \mintinline{c++}{Random(n+1)}. Если сгенерированное значение – $0$, поток аллоцирует еще один узел (вынимает из стека). Иначе – вызывает \mintinline{c++}{Free} на узле с полученным “случайным” индексом (возвращает в стек). Так потоки не будут привязаны к последовательности действий “\mintinline{c++}{Alloc} – \mintinline{c++}{Free} – \mintinline{c++}{Alloc} – \mintinline{c++}{Free} … “. Новому сценарию достаточно трех потоков вместо четырех для нарушения инварианта, время теста сокращается в шесть раз, а траектория получается короче – она не содержит лишних действий, появившихся из-за фиксированной структуры теста.

\iftoggleverb{pics}

\begin{figure}
	\centerfloat{
		\fbox{\includegraphics[scale=0.4]{stackrandom}}
	}
	\bigskip
	\caption{\mintinline{c++}{Random} в тесте lock-free стека.}
\end{figure}


\else

\begin{listing}
	\centering
	
	\begin{minted}{c++}
while (true) {
	size_t available = pool.ReleasedCount();
	int node_index = Random(available + 1);
	if (node_index == 0) {
		stack.Alloc();
	} else {
		Node* to_free = pool.Acquire(node_index - 1);
		stack.Free(to_free);
	}
}
	\end{minted}
	\caption{Random в тесте lock-free стека.}
	
\end{listing}

\fi 

Приведем окончательную статистику тестирования lock-free стека в LFAlloc:

\begin{table}
\centering
\begin{tabular}{| l | >{\bfseries}c | c | c |}
\hline
                              & \makecell{model checker \\ (с \mintinline{c++}{Random})}         & \makecell{model checker \\ (без \mintinline{c++}{Random})}  & TLC \\
                   \hline    
 Длина траектории             & 50 шагов                         & 61 шаг                      & 76 шагов \\ 
 Найдено состояний            & \numprint{2574023}               & \numprint{17906188}         & \numprint{26168885} \\      
 \makecell[l]{Найдено уникальных \\ состояний} & \numprint{789596}                & \numprint{4751588}          & \numprint{7269203} \\
 Время                        & 5 секунд                         & 29 cекунд                   & 8 минут\\
 	\hline
\end{tabular}

\bigskip
\captionsetup{justification=centering}
\caption{Статистика (с \mintinline{c++}{Random} и без) поиска ABA в LFAlloc.} 
\end{table}

Кроме того, функцию \mintinline{c++}{Random} использует сам model checker в реализации очереди ожидания в мьютексах и условных переменных: с помощью \mintinline{c++}{Random(n)} в \mintinline{c++}{WakeOne} он  может “недетерминированно” разбудить любой из $n$ спящих в очереди потоков.

\iftoggleverb{pics}

%\begin{comment}
	\begin{figure}
	\centerfloat{
		\fbox{\includegraphics[scale=0.5]{wakeone}}
	}
	\bigskip
	\caption{\mintinline{c++}{Random} в реализации очереди ожидания.}
\end{figure}
%\end{comment}

\else

\begin{listing}
\centering

\begin{minted}{c++}
void WaitQueue::WakeOne() {
	auto fibers = GetSuspendedFibers();
	if (fibers.empty()) {
		return;
	}
	size_t index = Random(fibers.size());
	Resume(fibers[index]);
}
\end{minted}
\caption{Random в реализации очереди ожидания.}

\end{listing}

\fi


\section{Prune}

Служебная функция \mintinline{c++}{Prune} выбрасывает из рассмотрения текущее состояние.

\mintinline{c++}{Prune} можно использовать для проверки тестов с бесконечным графом исполнения. Например, мы хотим написать тест для ticket lock-а – но атомарный счетчик свободных билетов растет в нем бесконечно. Чтобы ограничить граф состояний, вызовем \mintinline{c++}{Prune} в местах, где билет становится больше какого-то порога: 

\iftoggleverb{pics}

\begin{figure}
	\centerfloat{
		\fbox{\includegraphics[scale=0.5]{prune}}
	}
	\bigskip
	\caption{\mintinline{c++}{Prune} в ticket lock.}
\end{figure}

\else

\begin{listing}
	\centering
	
	\begin{minted}{c++}
void TicketLock::Lock() {
	size_t this_thread_ticket = next_free_ticket_.fetch_add(1);

	if (this_thread_ticket > kLimit) {
		Prune();
	}

	while (this_thread_ticket != owner_ticket_.load()) {
		// wait
	}
}
	\end{minted}
	\caption{Prune в ticket lock.}
	
\end{listing}

\fi

Еще одно потенциальное применение \mintinline{c++}{Prune} – отсечение неперспективных веток исполнения для ускорения перебора.

\section{Детализация траектории}


\subsection{Заметки о событиях}

Подобно логированию, траекторию model checker-а можно детализировать заметками с помощью макроса \mintinline{c++}{SHOW_NOTE(message)}.

Заметки делятся на два типа:

\begin{itemize}
\item	Служебные – заметки, уже встроенные в атомики / примитивы синхронизации: результат чтения атомика, парковка потока на условной переменной, состояние очереди ожидания мьютекса. 

\begin{figure}
	\centerfloat{
		\fbox{\includegraphics[scale=0.5]{spurcond}}
	}
	\bigskip
	\caption{Служебная заметка.}
\end{figure}

\item	Пользовательские – “отладочные” комментарии: поток зашел в определенную ветку кода, завершил какую-то операцию и т.п.

\begin{figure}
	\centerfloat{
		\fbox{\includegraphics[scale=0.45]{shownote}}
	}
	\bigskip
	\caption{Пользовательские заметки.}
\end{figure}

\end{itemize}

\subsection{Описание разделяемого состояния}

При печати траектории model checker автоматически показывает стеки вызовов и состояние локальных переменных. Но об устройстве глобального состояния он ничего не знает.

Чтобы каждый “шаг” траектории сопровождался представлением разделяемого состояния, пользователь должен в начале теста передать функцию печати model checker-у с помощью вызова \mintinline{c++}{PrintState(std::function<StateDescription()>)}.

Описание задается в виде \mintinline{c++}{StateDescription} – объекта, который model checker уже умеет печатать (в json-подобном виде). \mintinline{c++}{StateDescription} строится инкрементально двумя способами:

\begin{itemize}
\item	передачей пары “ключ – значение”;

\item	передачей вложенного \mintinline{c++}{StateDescription}-а.
\end{itemize}

\iftoggleverb{pics}

\begin{figure}
	\centerfloat{
		\fbox{\includegraphics[scale=0.35]{descr}}
	}
	\bigskip
	\caption{Описание разделяемого состояния в тесте спинлока.}\label{fig:descr}
\end{figure}

\else

\begin{listing}
	\centering
	
	\begin{minted}{c++}
PrintState([&lock, &in_critical_section]() {
	StateDescription d{"LockState"};
	d.Add(lock.DescribeState());
	d.Add("in_critical_section_", in_critical_section);
	return d;
});
	\end{minted}
	\caption{Описание разделяемого состояния.}
	
\end{listing}

\fi

\chapter{Заключение}



\section{Результаты}

Реализован model checker, который для многопоточного теста на C++ проверяет заданный инвариант во всех возможных состояниях, достижимых при исполнении этого теста. При нарушении инварианта model checker печатает кратчайшую детализированную траекторию, которая приводит к этому.

Model checker протестирован на нетривиальных примерах: он находит сложные баги из десятков шагов (ABA в lock-free стеке) и способен проверять сложный составной код (фреймворк экзекьюторов). 

Реализованы и описаны инструменты  для управления перебором в model checker-е: \mintinline{c++}{ForkGuarded<T>} для управления гранулярностью атомарности, \mintinline{c++}{Either} / \mintinline{c++}{Random} для выражения недетерминированного ветвления, \mintinline{c++}{Prune} для отсечения перебора. 

\section{Направления для дальнейшего исследования}

\begin{itemize}

\item	Поддержать слабые модели памяти.

\item	Автоматически инструментировать тестируемый код.

\end{itemize}



%\include{Dissertation/introduction}    % Введение
%\ifnumequal{\value{contnumfig}}{1}{\counterwithout{figure}{chapter}
%}{\counterwithin{figure}{chapter}}
%\ifnumequal{\value{contnumtab}}{1}{\counterwithout{table}{chapter}
%}{\counterwithin{table}{chapter}}
%\include{Dissertation/part1}           % Глава 1
%\include{Dissertation/part2}           % Глава 2
%\include{Dissertation/part3}           % Глава 3
%\include{Dissertation/conclusion}      % Заключение
%\include{Dissertation/acronyms}        % Список сокращений и условных обозначений
%\include{Dissertation/dictionary}      % Словарь терминов
\Urlmuskip=0mu plus 1mu
\raggedright
\include{Dissertation/references}      % Список литературы
%\clearpage
\ifdefmacro{\microtypesetup}{\microtypesetup{protrusion=false}}{} % не рекомендуется применять пакет микротипографики к автоматически генерируемым спискам

{
  \hypersetup{linkcolor=black}
  \listoffigures  % Список изображений
}
%%% Список таблиц %%%
% (ГОСТ Р 7.0.11-2011, 5.3.10)
\clearpage

{
  \hypersetup{linkcolor=black}
  \listoftables   % Список таблиц
}
\ifdefmacro{\microtypesetup}{\microtypesetup{protrusion=true}}{}
\newpage           % Списки таблиц и изображений (иллюстративный материал)

%%% Настройки для приложений
\appendix
% Оформление заголовков приложений ближе к ГОСТ:
\setlength{\midchapskip}{20pt}
\renewcommand*{\afterchapternum}{\par\nobreak\vskip \midchapskip}
\renewcommand\thechapter{\Asbuk{chapter}} % Чтобы приложения русскими буквами нумеровались

\chapter{LFAlloc}\label{app:A}

\newenvironment{codebr}{\captionsetup{type=listing}}{}

\section{Код аллокатора}


\begin{codebr}
	
	\begin{minted}[linenos, frame=leftline, fontsize=\tiny]{c++}
    
class LockFreeStack {
 public:
  void Push(Node* node) {
    Node* current_top = top_.load();
    do {
      node->next_.store(current_top);
    } while (!top_.compare_exchange_weak(current_top, node));
  }

  void Append(Node* head) {
    if (head == nullptr) {
      return;
    }

    Node* tail = head;
    Node* next = tail->next_.load();
    while (next != nullptr) {
      tail = next;
      next = tail->next_.load();
    }

    Node* prev_top = top_.load();
    do {
      tail->next_ = prev_top;
    } while (!top_.compare_exchange_weak(prev_top, head));
  }

  Node* Top() {
    return top_.load();
  }

  Node* Pop() {
    Node* curr_top = top_.load();
    Node* next_top;
    do {
      if (curr_top == nullptr) {
        break;
      }
      next_top = curr_top->next_.load();
    } while (!top_.compare_exchange_weak(curr_top, next_top));
    return curr_top;
  }

  bool TryClear(Node* expected) {
    Node* desired = nullptr;
    return top_.compare_exchange_strong(expected, desired);
  }

  StateDescription DescribeState(const std::string& name = "LFStack") {
    StateDescription d{name};
    auto head = top_.load();
    std::stringstream chain;
    while (head != nullptr) {
      chain << "-> " << head->GetLabel();
      head = head->next_;
    }
    d.Add("Blocks", chain.str());
    return d;
  }

 private:
  twist::stdlike::atomic<Node*> top_{nullptr};
};

//////////////////////////////////////////////////////////////////

class LFAllocator {
 public:
  LFAllocator(MemoryPool& pool) {
    twist::checker::ForkGuard guard;
    for (auto& node : pool.GetNodes()) {
      node.Acquire();
      item_list_.Push(&node);
    }
  }

  void Free(Node* node) {
    if (allocation_count_.load() == 0) {
      item_list_.Push(node);
    } else {
      pending_list_.Push(node);
    }
  }

  bool Alloc() {
    Node* pending_head = pending_list_.Top();

    if (allocation_count_.fetch_add(1) == 0) {
      if (pending_head != nullptr
          && pending_list_.TryClear(pending_head)) {
        auto head = pending_head;

        pending_head = pending_head->next_.load();
        head->Release();

        item_list_.Append(pending_head);
        allocation_count_.fetch_sub(1);

        return true;
      }
    }

    auto node = item_list_.Pop();
    if (node != nullptr) {
      node->Release();
    }

    allocation_count_.fetch_sub(1);

    return node != nullptr;
  }

  Node* Top() {
    return item_list_.Top();
  }

  Node* TopPending() {
    return item_list_.Top();
  }

  StateDescription DescribeState() {
    StateDescription d{"LFStack"};
    d.Add(item_list_.DescribeState("Head"));
    d.Add(pending_list_.DescribeState("Pending"));
    d.Add("allocation_count_", allocation_count_.load());
    return d;
  }

 private:
  LockFreeStack item_list_;
  LockFreeStack pending_list_;

  twist::stdlike::atomic<int> allocation_count_{0};
};

\end{minted}
%\caption{Код LFALLoc.}
	
\end{codebr}

\section{Вспомогательный код (валидация памяти)}

\begin{codebr}
	
\begin{minted}[linenos, frame=leftline, escapeinside=||, fontsize=\tiny]{c++}

struct Node {
  twist::stdlike::atomic<Node*> next_{nullptr};
  bool free_{true};
  std::string label_;

  bool IsReleased() const {
    return free_;
  }

  void Release() {
    free_ = true;
  }

  void Acquire() {
    free_ = false;
  }

  void SetLabel(const std::string& name) {
    std::ostringstream os;
    os << name << " (" << std::hex << this << std::dec << ") ";
    label_ = os.str();
  }

  std::string GetLabel() const {
    return label_;
  }
};

class MemoryPool {
 public:
  MemoryPool(size_t count) {
    nodes_ = std::vector<Node>{count};
    char name = 'A';

    for (auto it = nodes_.rbegin(); it != nodes_.rend(); ++it) {
      it->SetLabel(std::string(1, name++));
    }
  }

  std::vector<Node>& GetNodes() {
    return nodes_;
  }

  size_t ReleasedCount() const {
    size_t cnt = 0;
    for (auto& node : nodes_) {
      if (node.IsReleased()) {
        ++cnt;
      }
    }
    return cnt;
  }

  Node* Get(size_t index) {
    return &nodes_[index];
  }

  Node* Acquire(size_t index) {
    for (auto& node : nodes_) {
      if (node.IsReleased()) {
        if (index == 0) {
          node.Acquire();
          return &node;
        }
        |-{}-|index;
      }
    }
    TWIST_UNREACHABLE();
  }

  StateDescription DescribeState(const std::string& name = "Memory") {
    StateDescription description{name};
    for (auto& node : nodes_) {
      description.Add(node.GetLabel(), !node.free_);
    }
    return description;
  }

 private:
  std::vector<Node> nodes_;
};

\end{minted}
%\caption{Вспомогательный код.}
	
\end{codebr}

\section{Код теста}\label{app:A3}




\begin{codebr}
	
	\begin{minted}[linenos, frame=leftline, fontsize=\tiny]{c++}

TEST_SUITE(AllocLockFreeStack) {
  SIMPLE_CHECKER_TEST(Spec) {
    MemoryPool pool{3};
    LFAllocator stack{pool};

    Trace("/tmp/alloc_lfstack/aba.log");

    PrintState([&stack, &pool]() {
      StateDescription d{"LFAlloc"};
      d.Add(stack.DescribeState());
      d.Add(pool.DescribeState());
      return d;
    });

    AddInvariant([&stack]() -> std::pair<bool, std::string> {
      ForkGuard guard;

      auto check_list = [](Node* node) {
        while (node != nullptr) {
          if (node->IsReleased()) {
            return false;
          }
          node = node->next_.load();
        }
        return true;
      };

      if (!check_list(stack.Top())) {
        return {false, "ItemList: assertion failed"};
      }

      if (!check_list(stack.TopPending())) {
        return {false, "PendingList: assertion failed"};
      }

      return {true, ""};
    }).ExpectFail();

    auto worker = [&stack, &pool]() {
      while (true) {
        size_t available = pool.ReleasedCount();
        int node_index = Random(available + 1);
        if (node_index == 0) {
          SHOW_NOTE("Alloc(): before");
          stack.Alloc();
          SHOW_NOTE("Alloc(): after");
        } else {
          SHOW_NOTE("Free(): before");
          Node* to_free = pool.Acquire(node_index - 1);
          stack.Free(to_free);
          SHOW_NOTE("Free(): after");
        }
      }
    };

    std::vector<twist::stdlike::thread> threads;
    for (size_t i = 0; i < 3; ++i) {
      threads.emplace_back(worker);
    }

    for (auto& th : threads) {
      th.join();
    }
  }
}

	\end{minted}
	%\caption{Тест LFALLoc.}
	
\end{codebr}

\pagebreak

\section{Траектория}\label{app:trace}

\begin{adjustwidth}{-3.3cm}{-1cm}

\definecolor{yellow}{rgb}{1.0, 0.75, 0.0}

%\newfontfamily\myfont[<font features>]{<name of font>}

\newenvironment{allintypewriter}{\ttfamily}{\par}

%\setlength{\columnseprule}{0.5pt}
\def\columnseprulecolor{\color{gray}}
\setlength\columnsep{0pt}

\begin{allintypewriter}
%\fontfamily{ptm}\selectfont 

\noindent
\obeyspaces
\begin{multicols*}{2}
\noindent	
\obeyspaces
\begin{lstlisting}[numbers=none]

<@\textcolor{yellow}{==========================================================================================}@>
                                 <@\textcolor{yellow}{THREAD SWITCHING HISTORY}@>
<@\textcolor{yellow}{==========================================================================================}@>

0 1 1 1 1 1 1 21 2 1 1 2 2 2 2 2 3 3 2 2 20 2 3 1 1 1 2 2 1 11 1 1 2 1 11 1 1 1 1 1 1 1 1 
1 3 3 3 3 3 3 2 

<@\textcolor{yellow}{==========================================================================================}@>
                                     <@\textcolor{yellow}{RUN 1 (THREAD 0)}@>
<@\textcolor{yellow}{==========================================================================================}@>

<@\textcolor{cyan}{------------------------------------------------------------------------------------------}@>
                                          <@\textcolor{cyan}{BEFORE}@>
<@\textcolor{cyan}{------------------------------------------------------------------------------------------}@>

STATE: STARTING

<@\textcolor{cyan}{------------------------------------------------------------------------------------------}@>
                                         <@\textcolor{cyan}{RUNNING}@>
<@\textcolor{cyan}{------------------------------------------------------------------------------------------}@>

Joining thread 0

<@\textcolor{cyan}{------------------------------------------------------------------------------------------}@>
                                          <@\textcolor{cyan}{AFTER}@>
<@\textcolor{cyan}{------------------------------------------------------------------------------------------}@>

STATE: RUNNING -> SUSPENDED

Blocked by event awaiting (wait queue: 0)

<@\textcolor{cyan}{------------------------------------------------------------------------------------------}@>

<@\textcolor{yellow}{==========================================================================================}@>
                                     <@\textcolor{yellow}{RUN 2 (THREAD 1)}@>
<@\textcolor{yellow}{==========================================================================================}@>

<@\textcolor{cyan}{------------------------------------------------------------------------------------------}@>
                                          <@\textcolor{cyan}{BEFORE}@>
<@\textcolor{cyan}{------------------------------------------------------------------------------------------}@>

STATE: STARTING

<@\textcolor{cyan}{------------------------------------------------------------------------------------------}@>
                                         <@\textcolor{cyan}{RUNNING}@>
<@\textcolor{cyan}{------------------------------------------------------------------------------------------}@>

Alloc(): before

<@\textcolor{cyan}{------------------------------------------------------------------------------------------}@>
                                        <@\textcolor{cyan}{SNAPSHOT 1}@>
<@\textcolor{cyan}{------------------------------------------------------------------------------------------}@>

<@\textcolor{red}{SHARED STATE:}@>

    LFAlloc {
	    LFStack {
		    Head {
			    Blocks: -> A (0x7fc21bc96070) -> B (0x7fc21bc96040) -> C (0x7fc21bc96010) 
		    }

		    Pending {
			    Blocks: 
		    }

		    allocation_count_: 0
	    }

	    Memory {
		    C (0x7fc21bc96010) : 1
		    B (0x7fc21bc96040) : 1
		    A (0x7fc21bc96070) : 1
	    }

    }

<@\textcolor{cyan}{------------------------------------------------------------------------------------------}@>
<@\textcolor{magenta}{
STACK:
}@>
Object "/home/sham42/git/twist-mc-test/tests/alloc_lfstack3/alloc_lfstack.hpp",
in LockFreeStack::Top()
  38:   Node* Top() {
> <@\textcolor{magenta}{39:     return top\_.load();}@>
  40:   }

<@\textcolor{cyan}{----------------}@>

Object "/home/sham42/git/twist-mc-test/tests/alloc_lfstack3/alloc_lfstack.hpp",
in LFAllocator::Alloc()
  96:     // auto keep_count = pending_count_.load();
> <@\textcolor{magenta}{97:     Node* pending\_head = pending\_list\_.Top();}@>

Local variables: 
pending_head = 0x7fc21bc8cea0
node = 0x7ffd4ad1d480

<@\textcolor{cyan}{----------------}@>

Object "/home/sham42/git/twist-mc-test/tests/alloc_lfstack3/test.cpp",
in operator()
  61:           SHOW_NOTE("Alloc(): before");
> <@\textcolor{magenta}{62:           stack.Alloc();}@>
  63:           SHOW_NOTE("Alloc(): after");

Local variables: 
available = 0x0
node_index = 0xffffffff

<@\textcolor{cyan}{------------------------------------------------------------------------------------------}@>
                                         <@\textcolor{cyan}{RUNNING}@>
<@\textcolor{cyan}{------------------------------------------------------------------------------------------}@>

load(): 0

<@\textcolor{cyan}{------------------------------------------------------------------------------------------}@>
                                        <@\textcolor{cyan}{SNAPSHOT 2}@>
<@\textcolor{cyan}{------------------------------------------------------------------------------------------}@>

<@\textcolor{red}{SHARED STATE:}@>

    LFAlloc {
	    LFStack {
		    Head {
			    Blocks: -> A (0x7fc21bc96070) -> B (0x7fc21bc96040) -> C (0x7fc21bc96010) 
		    }

		    Pending {
			    Blocks: 
		    }

		    allocation_count_: 0
	    }

	    Memory {
		    C (0x7fc21bc96010) : 1
		    B (0x7fc21bc96040) : 1
		    A (0x7fc21bc96070) : 1
	    }

    }

<@\textcolor{cyan}{------------------------------------------------------------------------------------------}@>
<@\textcolor{magenta}{
STACK:
}@>
Object "/home/sham42/git/twist-mc-test/tests/alloc_lfstack3/alloc_lfstack.hpp",
in LFAllocator::Alloc()
> <@\textcolor{magenta}{99:     if (allocation\_count\_.fetch\_add(1) == 0) \{}@>
  100:       if (pending_head != nullptr /*&& keep_count == pending_count_.load()*/

Local variables: 
pending_head = 0x0
node = 0x7ffd4ad1d480

<@\textcolor{cyan}{----------------}@>

Object "/home/sham42/git/twist-mc-test/tests/alloc_lfstack3/test.cpp",
in operator()
  61:           SHOW_NOTE("Alloc(): before");
> <@\textcolor{magenta}{62:           stack.Alloc();}@>
  63:           SHOW_NOTE("Alloc(): after");

Local variables: 
available = 0x0
node_index = 0xffffffff

<@\textcolor{cyan}{------------------------------------------------------------------------------------------}@>
                                         <@\textcolor{cyan}{RUNNING}@>
<@\textcolor{cyan}{------------------------------------------------------------------------------------------}@>

fetch_add: prev = 0

<@\textcolor{cyan}{------------------------------------------------------------------------------------------}@>
                                        <@\textcolor{cyan}{SNAPSHOT 3}@>
<@\textcolor{cyan}{------------------------------------------------------------------------------------------}@>

<@\textcolor{red}{SHARED STATE:}@>

    LFAlloc {
	    LFStack {
		    Head {
			    Blocks: -> A (0x7fc21bc96070) -> B (0x7fc21bc96040) -> C (0x7fc21bc96010) 
		    }

		    Pending {
			    Blocks: 
		    }

		<@\textcolor{red}{[*] allocation\_count\_: 1}@>
	    }

	    Memory {
		    C (0x7fc21bc96010) : 1
		    B (0x7fc21bc96040) : 1
		    A (0x7fc21bc96070) : 1
	    }

    }

<@\textcolor{cyan}{------------------------------------------------------------------------------------------}@>
<@\textcolor{magenta}{
STACK:
}@>
Object "/home/sham42/git/twist-mc-test/tests/alloc_lfstack3/alloc_lfstack.hpp",
in LockFreeStack::Pop()
  42:   Node* Pop() {
> <@\textcolor{magenta}{43:     Node* curr\_top = top\_.load();}@>
  44:     Node* next_top;

Local variables: 
curr_top = 0x100000005
next_top = 0x100000000

<@\textcolor{cyan}{----------------}@>

Object "/home/sham42/git/twist-mc-test/tests/alloc_lfstack3/alloc_lfstack.hpp",
in LFAllocator::Alloc()
> <@\textcolor{magenta}{115:     auto node = item\_list\_.Pop();}@>
  116:     if (node != nullptr) {

Local variables: 
pending_head = 0x0
node = 0x7ffd4ad1d480

<@\textcolor{cyan}{----------------}@>

Object "/home/sham42/git/twist-mc-test/tests/alloc_lfstack3/test.cpp",
in operator()
  61:           SHOW_NOTE("Alloc(): before");
> <@\textcolor{magenta}{62:           stack.Alloc();}@>
  63:           SHOW_NOTE("Alloc(): after");

Local variables: 
available = 0x0
node_index = 0xffffffff

<@\textcolor{cyan}{------------------------------------------------------------------------------------------}@>
                                         <@\textcolor{cyan}{RUNNING}@>
<@\textcolor{cyan}{------------------------------------------------------------------------------------------}@>

load(): 0x7fc21bc96070

<@\textcolor{cyan}{------------------------------------------------------------------------------------------}@>
                                        <@\textcolor{cyan}{SNAPSHOT 4}@>
<@\textcolor{cyan}{------------------------------------------------------------------------------------------}@>

<@\textcolor{red}{SHARED STATE:}@>

    LFAlloc {
	    LFStack {
		    Head {
			    Blocks: -> A (0x7fc21bc96070) -> B (0x7fc21bc96040) -> C (0x7fc21bc96010) 
		    }

		    Pending {
			    Blocks: 
		    }

		    allocation_count_: 1
	    }

	    Memory {
		    C (0x7fc21bc96010) : 1
		    B (0x7fc21bc96040) : 1
		    A (0x7fc21bc96070) : 1
	    }

    }

<@\textcolor{cyan}{------------------------------------------------------------------------------------------}@>
<@\textcolor{magenta}{
STACK:
}@>
Object "/home/sham42/git/twist-mc-test/tests/alloc_lfstack3/alloc_lfstack.hpp",
in LockFreeStack::Pop()
  48:       }
> <@\textcolor{magenta}{49:       next\_top = curr\_top->next\_.load();}@>
  50:     } while (!top_.compare_exchange_weak(curr_top, next_top));

Local variables: 
curr_top = 0x7fc21bc96070
next_top = 0x100000000

<@\textcolor{cyan}{----------------}@>

Object "/home/sham42/git/twist-mc-test/tests/alloc_lfstack3/alloc_lfstack.hpp",
in LFAllocator::Alloc()
> <@\textcolor{magenta}{115:     auto node = item\_list\_.Pop();}@>
  116:     if (node != nullptr) {

Local variables: 
pending_head = 0x0
node = 0x7ffd4ad1d480

<@\textcolor{cyan}{----------------}@>

Object "/home/sham42/git/twist-mc-test/tests/alloc_lfstack3/test.cpp",
in operator()
  61:           SHOW_NOTE("Alloc(): before");
> <@\textcolor{magenta}{62:           stack.Alloc();}@>
  63:           SHOW_NOTE("Alloc(): after");

Local variables: 
available = 0x0
node_index = 0xffffffff

<@\textcolor{cyan}{------------------------------------------------------------------------------------------}@>
                                         <@\textcolor{cyan}{RUNNING}@>
<@\textcolor{cyan}{------------------------------------------------------------------------------------------}@>

load(): 0x7fc21bc96040

<@\textcolor{cyan}{------------------------------------------------------------------------------------------}@>
                                        <@\textcolor{cyan}{SNAPSHOT 5}@>
<@\textcolor{cyan}{------------------------------------------------------------------------------------------}@>

<@\textcolor{red}{SHARED STATE:}@>

    LFAlloc {
	    LFStack {
		    Head {
			    Blocks: -> A (0x7fc21bc96070) -> B (0x7fc21bc96040) -> C (0x7fc21bc96010) 
		    }

		    Pending {
			    Blocks: 
		    }

		    allocation_count_: 1
	    }

	    Memory {
		    C (0x7fc21bc96010) : 1
		    B (0x7fc21bc96040) : 1
		    A (0x7fc21bc96070) : 1
	    }

    }

<@\textcolor{cyan}{------------------------------------------------------------------------------------------}@>
<@\textcolor{magenta}{
STACK:
}@>
Object "/home/sham42/git/twist-mc-test/tests/alloc_lfstack3/alloc_lfstack.hpp",
in LockFreeStack::Pop()
  49:       next_top = curr_top->next_.load();
> <@\textcolor{magenta}{50:     \} while (!top\_.compare\_exchange\_weak(curr\_top, next\_top));}@>
  51:     return curr_top;

Local variables: 
curr_top = 0x7fc21bc96070
next_top = 0x7fc21bc96040

<@\textcolor{cyan}{----------------}@>

Object "/home/sham42/git/twist-mc-test/tests/alloc_lfstack3/alloc_lfstack.hpp",
in LFAllocator::Alloc()
> <@\textcolor{magenta}{115:     auto node = item\_list\_.Pop();}@>
  116:     if (node != nullptr) {

Local variables: 
pending_head = 0x0
node = 0x7ffd4ad1d480

<@\textcolor{cyan}{----------------}@>

Object "/home/sham42/git/twist-mc-test/tests/alloc_lfstack3/test.cpp",
in operator()
  61:           SHOW_NOTE("Alloc(): before");
> <@\textcolor{magenta}{62:           stack.Alloc();}@>
  63:           SHOW_NOTE("Alloc(): after");

Local variables: 
available = 0x0
node_index = 0xffffffff

<@\textcolor{cyan}{------------------------------------------------------------------------------------------}@>
                                         <@\textcolor{cyan}{RUNNING}@>
<@\textcolor{cyan}{------------------------------------------------------------------------------------------}@>

compare_exchange_weak(): succeeded = 1

<@\textcolor{cyan}{------------------------------------------------------------------------------------------}@>
                                          <@\textcolor{cyan}{AFTER}@>
<@\textcolor{cyan}{------------------------------------------------------------------------------------------}@>

<@\textcolor{red}{SHARED STATE:}@>

    LFAlloc {
	    LFStack {
		    Head {
			<@\textcolor{red}{[*] Blocks: -> B (0x7fc21bc96040) -> C (0x7fc21bc96010) }@>
		    }

		    Pending {
			    Blocks: 
		    }

		    allocation_count_: 1
	    }

	    Memory {
		    C (0x7fc21bc96010) : 1
		    B (0x7fc21bc96040) : 1
		<@\textcolor{red}{[*] A (0x7fc21bc96070) : 0}@>
	    }

    }

<@\textcolor{cyan}{------------------------------------------------------------------------------------------}@>
<@\textcolor{magenta}{
STACK:
}@>
Object "/home/sham42/git/twist-mc-test/tests/alloc_lfstack3/alloc_lfstack.hpp",
in LFAllocator::Alloc()
> <@\textcolor{magenta}{120:     allocation\_count\_.fetch\_sub(1);}@>

Local variables: 
pending_head = 0x0
node = 0x7fc21bc96070

<@\textcolor{cyan}{----------------}@>

Object "/home/sham42/git/twist-mc-test/tests/alloc_lfstack3/test.cpp",
in operator()
  61:           SHOW_NOTE("Alloc(): before");
> <@\textcolor{magenta}{62:           stack.Alloc();}@>
  63:           SHOW_NOTE("Alloc(): after");

Local variables: 
available = 0x0
node_index = 0xffffffff

<@\textcolor{yellow}{==========================================================================================}@>
                                     <@\textcolor{yellow}{RUN 3 (THREAD 2)}@>
<@\textcolor{yellow}{==========================================================================================}@>

<@\textcolor{cyan}{------------------------------------------------------------------------------------------}@>
                                          <@\textcolor{cyan}{BEFORE}@>
<@\textcolor{cyan}{------------------------------------------------------------------------------------------}@>

STATE: STARTING

<@\textcolor{cyan}{------------------------------------------------------------------------------------------}@>
                                         <@\textcolor{cyan}{RUNNING}@>
<@\textcolor{cyan}{------------------------------------------------------------------------------------------}@>

Free(): before

<@\textcolor{cyan}{------------------------------------------------------------------------------------------}@>
                                        <@\textcolor{cyan}{SNAPSHOT 1}@>
<@\textcolor{cyan}{------------------------------------------------------------------------------------------}@>

<@\textcolor{red}{SHARED STATE:}@>

    LFAlloc {
	    LFStack {
		    Head {
			    Blocks: -> B (0x7fc21bc96040) -> C (0x7fc21bc96010) 
		    }

		    Pending {
			    Blocks: 
		    }

		    allocation_count_: 1
	    }

	    Memory {
		    C (0x7fc21bc96010) : 1
		    B (0x7fc21bc96040) : 1
		<@\textcolor{red}{[*] A (0x7fc21bc96070) : 1}@>
	    }

    }

<@\textcolor{cyan}{------------------------------------------------------------------------------------------}@>
<@\textcolor{magenta}{
STACK:
}@>
Object "/home/sham42/git/twist-mc-test/tests/alloc_lfstack3/alloc_lfstack.hpp",
in LFAllocator::Free(Node*)
  87:   void Free(Node* node) /*__attribute__((noinline))*/ {
> <@\textcolor{magenta}{88:     if (allocation\_count\_.load() == 0) \{}@>
  89:       item_list_.Push(node);

<@\textcolor{cyan}{----------------}@>

Object "/home/sham42/git/twist-mc-test/tests/alloc_lfstack3/test.cpp",
in operator()
  66:           Node* to_free = pool.Acquire(node_index);
> <@\textcolor{magenta}{67:           stack.Free(to\_free);}@>
  68:           SHOW_NOTE("Free(): after");

Local variables: 
to_free = 0x7fc21bc96070
available = 0x1
node_index = 0x0

<@\textcolor{cyan}{------------------------------------------------------------------------------------------}@>
                                         <@\textcolor{cyan}{RUNNING}@>
<@\textcolor{cyan}{------------------------------------------------------------------------------------------}@>

load(): 1

<@\textcolor{cyan}{------------------------------------------------------------------------------------------}@>
                                          <@\textcolor{cyan}{AFTER}@>
<@\textcolor{cyan}{------------------------------------------------------------------------------------------}@>

<@\textcolor{red}{SHARED STATE:}@>

    LFAlloc {
	    LFStack {
		    Head {
			    Blocks: -> B (0x7fc21bc96040) -> C (0x7fc21bc96010) 
		    }

		    Pending {
			    Blocks: 
		    }

		    allocation_count_: 1
	    }

	    Memory {
		    C (0x7fc21bc96010) : 1
		    B (0x7fc21bc96040) : 1
		    A (0x7fc21bc96070) : 1
	    }

    }

<@\textcolor{cyan}{------------------------------------------------------------------------------------------}@>
<@\textcolor{magenta}{
STACK:
}@>
Object "/home/sham42/git/twist-mc-test/tests/alloc_lfstack3/alloc_lfstack.hpp",
in LockFreeStack::Push(Node*)
  13:   void Push(Node* node) {
> <@\textcolor{magenta}{14:     Node* current\_top = top\_.load();}@>
  15:     do {

Local variables: 
current_top = 0x100413133

<@\textcolor{cyan}{----------------}@>

Object "/home/sham42/git/twist-mc-test/tests/alloc_lfstack3/alloc_lfstack.hpp",
in LFAllocator::Free(Node*)
  90:     } else {
> <@\textcolor{magenta}{91:       pending\_list\_.Push(node);}@>
  92:     }

<@\textcolor{cyan}{----------------}@>

Object "/home/sham42/git/twist-mc-test/tests/alloc_lfstack3/test.cpp",
in operator()
  66:           Node* to_free = pool.Acquire(node_index);
> <@\textcolor{magenta}{67:           stack.Free(to\_free);}@>
  68:           SHOW_NOTE("Free(): after");

Local variables: 
to_free = 0x7fc21bc96070
available = 0x1
node_index = 0x0

<@\textcolor{yellow}{==========================================================================================}@>
                                     <@\textcolor{yellow}{RUN 4 (THREAD 1)}@>
<@\textcolor{yellow}{==========================================================================================}@>

<@\textcolor{cyan}{------------------------------------------------------------------------------------------}@>
                                          <@\textcolor{cyan}{BEFORE}@>
<@\textcolor{cyan}{------------------------------------------------------------------------------------------}@>
<@\textcolor{magenta}{
STACK:
}@>
Object "/home/sham42/git/twist-mc-test/tests/alloc_lfstack3/alloc_lfstack.hpp",
in LFAllocator::Alloc()
> <@\textcolor{magenta}{120:     allocation\_count\_.fetch\_sub(1);}@>

Local variables: 
pending_head = 0x0
node = 0x7fc21bc96070

<@\textcolor{cyan}{----------------}@>

Object "/home/sham42/git/twist-mc-test/tests/alloc_lfstack3/test.cpp",
in operator()
  61:           SHOW_NOTE("Alloc(): before");
> <@\textcolor{magenta}{62:           stack.Alloc();}@>
  63:           SHOW_NOTE("Alloc(): after");

Local variables: 
available = 0x0
node_index = 0xffffffff

<@\textcolor{cyan}{------------------------------------------------------------------------------------------}@>
                                         <@\textcolor{cyan}{RUNNING}@>
<@\textcolor{cyan}{------------------------------------------------------------------------------------------}@>

fetch_sub(): prev = 1

Alloc(): after

Alloc(): before

<@\textcolor{cyan}{------------------------------------------------------------------------------------------}@>
                                        <@\textcolor{cyan}{SNAPSHOT 1}@>
<@\textcolor{cyan}{------------------------------------------------------------------------------------------}@>

<@\textcolor{red}{SHARED STATE:}@>

    LFAlloc {
	    LFStack {
		    Head {
			    Blocks: -> B (0x7fc21bc96040) -> C (0x7fc21bc96010) 
		    }

		    Pending {
			    Blocks: 
		    }

		<@\textcolor{red}{[*] allocation\_count\_: 0}@>
	    }

	    Memory {
		    C (0x7fc21bc96010) : 1
		    B (0x7fc21bc96040) : 1
		    A (0x7fc21bc96070) : 1
	    }

    }

<@\textcolor{cyan}{------------------------------------------------------------------------------------------}@>
<@\textcolor{magenta}{
STACK:
}@>
Object "/home/sham42/git/twist-mc-test/tests/alloc_lfstack3/alloc_lfstack.hpp",
in LockFreeStack::Top()
  38:   Node* Top() {
> <@\textcolor{magenta}{39:     return top\_.load();}@>
  40:   }

<@\textcolor{cyan}{----------------}@>

Object "/home/sham42/git/twist-mc-test/tests/alloc_lfstack3/alloc_lfstack.hpp",
in LFAllocator::Alloc()
  96:     // auto keep_count = pending_count_.load();
> <@\textcolor{magenta}{97:     Node* pending\_head = pending\_list\_.Top();}@>

Local variables: 
pending_head = 0x7fc21bc8cea0
node = 0x7ffd4ad1d480

<@\textcolor{cyan}{----------------}@>

Object "/home/sham42/git/twist-mc-test/tests/alloc_lfstack3/test.cpp",
in operator()
  61:           SHOW_NOTE("Alloc(): before");
> <@\textcolor{magenta}{62:           stack.Alloc();}@>
  63:           SHOW_NOTE("Alloc(): after");

Local variables: 
available = 0x0
node_index = 0xffffffff

<@\textcolor{cyan}{------------------------------------------------------------------------------------------}@>
                                         <@\textcolor{cyan}{RUNNING}@>
<@\textcolor{cyan}{------------------------------------------------------------------------------------------}@>

load(): 0

<@\textcolor{cyan}{------------------------------------------------------------------------------------------}@>
                                          <@\textcolor{cyan}{AFTER}@>
<@\textcolor{cyan}{------------------------------------------------------------------------------------------}@>

<@\textcolor{red}{SHARED STATE:}@>

    LFAlloc {
	    LFStack {
		    Head {
			    Blocks: -> B (0x7fc21bc96040) -> C (0x7fc21bc96010) 
		    }

		    Pending {
			    Blocks: 
		    }

		    allocation_count_: 0
	    }

	    Memory {
		    C (0x7fc21bc96010) : 1
		    B (0x7fc21bc96040) : 1
		    A (0x7fc21bc96070) : 1
	    }

    }

<@\textcolor{cyan}{------------------------------------------------------------------------------------------}@>
<@\textcolor{magenta}{
STACK:
}@>
Object "/home/sham42/git/twist-mc-test/tests/alloc_lfstack3/alloc_lfstack.hpp",
in LFAllocator::Alloc()
> <@\textcolor{magenta}{99:     if (allocation\_count\_.fetch\_add(1) == 0) \{}@>
  100:       if (pending_head != nullptr /*&& keep_count == pending_count_.load()*/

Local variables: 
pending_head = 0x0
node = 0x7ffd4ad1d480

<@\textcolor{cyan}{----------------}@>

Object "/home/sham42/git/twist-mc-test/tests/alloc_lfstack3/test.cpp",
in operator()
  61:           SHOW_NOTE("Alloc(): before");
> <@\textcolor{magenta}{62:           stack.Alloc();}@>
  63:           SHOW_NOTE("Alloc(): after");

Local variables: 
available = 0x0
node_index = 0xffffffff

<@\textcolor{yellow}{==========================================================================================}@>
                                     <@\textcolor{yellow}{RUN 5 (THREAD 2)}@>
<@\textcolor{yellow}{==========================================================================================}@>

<@\textcolor{cyan}{------------------------------------------------------------------------------------------}@>
                                          <@\textcolor{cyan}{BEFORE}@>
<@\textcolor{cyan}{------------------------------------------------------------------------------------------}@>
<@\textcolor{magenta}{
STACK:
}@>
Object "/home/sham42/git/twist-mc-test/tests/alloc_lfstack3/alloc_lfstack.hpp",
in LockFreeStack::Push(Node*)
  13:   void Push(Node* node) {
> <@\textcolor{magenta}{14:     Node* current\_top = top\_.load();}@>
  15:     do {

Local variables: 
current_top = 0x100413133

<@\textcolor{cyan}{----------------}@>

Object "/home/sham42/git/twist-mc-test/tests/alloc_lfstack3/alloc_lfstack.hpp",
in LFAllocator::Free(Node*)
  90:     } else {
> <@\textcolor{magenta}{91:       pending\_list\_.Push(node);}@>
  92:     }

<@\textcolor{cyan}{----------------}@>

Object "/home/sham42/git/twist-mc-test/tests/alloc_lfstack3/test.cpp",
in operator()
  66:           Node* to_free = pool.Acquire(node_index);
> <@\textcolor{magenta}{67:           stack.Free(to\_free);}@>
  68:           SHOW_NOTE("Free(): after");

Local variables: 
to_free = 0x7fc21bc96070
available = 0x1
node_index = 0x0

<@\textcolor{cyan}{------------------------------------------------------------------------------------------}@>
                                         <@\textcolor{cyan}{RUNNING}@>
<@\textcolor{cyan}{------------------------------------------------------------------------------------------}@>

load(): 0

<@\textcolor{cyan}{------------------------------------------------------------------------------------------}@>
                                        <@\textcolor{cyan}{SNAPSHOT 1}@>
<@\textcolor{cyan}{------------------------------------------------------------------------------------------}@>

<@\textcolor{red}{SHARED STATE:}@>

    LFAlloc {
	    LFStack {
		    Head {
			    Blocks: -> B (0x7fc21bc96040) -> C (0x7fc21bc96010) 
		    }

		    Pending {
			    Blocks: 
		    }

		    allocation_count_: 0
	    }

	    Memory {
		    C (0x7fc21bc96010) : 1
		    B (0x7fc21bc96040) : 1
		    A (0x7fc21bc96070) : 1
	    }

    }

<@\textcolor{cyan}{------------------------------------------------------------------------------------------}@>
<@\textcolor{magenta}{
STACK:
}@>
Object "/home/sham42/git/twist-mc-test/tests/alloc_lfstack3/alloc_lfstack.hpp",
in LockFreeStack::Push(Node*)
  15:     do {
> <@\textcolor{magenta}{16:       node->next\_.store(current\_top);}@>
  17:     } while (!top_.compare_exchange_weak(current_top, node));

Local variables: 
current_top = 0x0

<@\textcolor{cyan}{----------------}@>

Object "/home/sham42/git/twist-mc-test/tests/alloc_lfstack3/alloc_lfstack.hpp",
in LFAllocator::Free(Node*)
  90:     } else {
> <@\textcolor{magenta}{91:       pending\_list\_.Push(node);}@>
  92:     }

<@\textcolor{cyan}{----------------}@>

Object "/home/sham42/git/twist-mc-test/tests/alloc_lfstack3/test.cpp",
in operator()
  66:           Node* to_free = pool.Acquire(node_index);
> <@\textcolor{magenta}{67:           stack.Free(to\_free);}@>
  68:           SHOW_NOTE("Free(): after");

Local variables: 
to_free = 0x7fc21bc96070
available = 0x1
node_index = 0x0

<@\textcolor{cyan}{------------------------------------------------------------------------------------------}@>
                                        <@\textcolor{cyan}{SNAPSHOT 2}@>
<@\textcolor{cyan}{------------------------------------------------------------------------------------------}@>

<@\textcolor{red}{SHARED STATE:}@>

    LFAlloc {
	    LFStack {
		    Head {
			    Blocks: -> B (0x7fc21bc96040) -> C (0x7fc21bc96010) 
		    }

		    Pending {
			    Blocks: 
		    }

		    allocation_count_: 0
	    }

	    Memory {
		    C (0x7fc21bc96010) : 1
		    B (0x7fc21bc96040) : 1
		    A (0x7fc21bc96070) : 1
	    }

    }

<@\textcolor{cyan}{------------------------------------------------------------------------------------------}@>
<@\textcolor{magenta}{
STACK:
}@>
Object "/home/sham42/git/twist-mc-test/tests/alloc_lfstack3/alloc_lfstack.hpp",
in LockFreeStack::Push(Node*)
  16:       node->next_.store(current_top);
> <@\textcolor{magenta}{17:     \} while (!top\_.compare\_exchange\_weak(current\_top, node));}@>
  18:   }

Local variables: 
current_top = 0x0

<@\textcolor{cyan}{----------------}@>

Object "/home/sham42/git/twist-mc-test/tests/alloc_lfstack3/alloc_lfstack.hpp",
in LFAllocator::Free(Node*)
  90:     } else {
> <@\textcolor{magenta}{91:       pending\_list\_.Push(node);}@>
  92:     }

<@\textcolor{cyan}{----------------}@>

Object "/home/sham42/git/twist-mc-test/tests/alloc_lfstack3/test.cpp",
in operator()
  66:           Node* to_free = pool.Acquire(node_index);
> <@\textcolor{magenta}{67:           stack.Free(to\_free);}@>
  68:           SHOW_NOTE("Free(): after");

Local variables: 
to_free = 0x7fc21bc96070
available = 0x1
node_index = 0x0

<@\textcolor{cyan}{------------------------------------------------------------------------------------------}@>
                                         <@\textcolor{cyan}{RUNNING}@>
<@\textcolor{cyan}{------------------------------------------------------------------------------------------}@>

compare_exchange_weak(): succeeded = 1

Free(): after

Alloc(): before

<@\textcolor{cyan}{------------------------------------------------------------------------------------------}@>
                                        <@\textcolor{cyan}{SNAPSHOT 3}@>
<@\textcolor{cyan}{------------------------------------------------------------------------------------------}@>

<@\textcolor{red}{SHARED STATE:}@>

    LFAlloc {
	    LFStack {
		    Head {
			    Blocks: -> B (0x7fc21bc96040) -> C (0x7fc21bc96010) 
		    }

		    Pending {
			<@\textcolor{red}{[*] Blocks: -> A (0x7fc21bc96070) }@>
		    }

		    allocation_count_: 0
	    }

	    Memory {
		    C (0x7fc21bc96010) : 1
		    B (0x7fc21bc96040) : 1
		    A (0x7fc21bc96070) : 1
	    }

    }

<@\textcolor{cyan}{------------------------------------------------------------------------------------------}@>
<@\textcolor{magenta}{
STACK:
}@>
Object "/home/sham42/git/twist-mc-test/tests/alloc_lfstack3/alloc_lfstack.hpp",
in LockFreeStack::Top()
  38:   Node* Top() {
> <@\textcolor{magenta}{39:     return top\_.load();}@>
  40:   }

<@\textcolor{cyan}{----------------}@>

Object "/home/sham42/git/twist-mc-test/tests/alloc_lfstack3/alloc_lfstack.hpp",
in LFAllocator::Alloc()
  96:     // auto keep_count = pending_count_.load();
> <@\textcolor{magenta}{97:     Node* pending\_head = pending\_list\_.Top();}@>

Local variables: 
pending_head = 0x7fb80b3baea0
node = 0x7ffd4ad1d480

<@\textcolor{cyan}{----------------}@>

Object "/home/sham42/git/twist-mc-test/tests/alloc_lfstack3/test.cpp",
in operator()
  61:           SHOW_NOTE("Alloc(): before");
> <@\textcolor{magenta}{62:           stack.Alloc();}@>
  63:           SHOW_NOTE("Alloc(): after");

Local variables: 
available = 0x0
node_index = 0xffffffff

<@\textcolor{cyan}{------------------------------------------------------------------------------------------}@>
                                         <@\textcolor{cyan}{RUNNING}@>
<@\textcolor{cyan}{------------------------------------------------------------------------------------------}@>

load(): 0x7fc21bc96070

<@\textcolor{cyan}{------------------------------------------------------------------------------------------}@>
                                        <@\textcolor{cyan}{SNAPSHOT 4}@>
<@\textcolor{cyan}{------------------------------------------------------------------------------------------}@>

<@\textcolor{red}{SHARED STATE:}@>

    LFAlloc {
	    LFStack {
		    Head {
			    Blocks: -> B (0x7fc21bc96040) -> C (0x7fc21bc96010) 
		    }

		    Pending {
			    Blocks: -> A (0x7fc21bc96070) 
		    }

		    allocation_count_: 0
	    }

	    Memory {
		    C (0x7fc21bc96010) : 1
		    B (0x7fc21bc96040) : 1
		    A (0x7fc21bc96070) : 1
	    }

    }

<@\textcolor{cyan}{------------------------------------------------------------------------------------------}@>
<@\textcolor{magenta}{
STACK:
}@>
Object "/home/sham42/git/twist-mc-test/tests/alloc_lfstack3/alloc_lfstack.hpp",
in LFAllocator::Alloc()
> <@\textcolor{magenta}{99:     if (allocation\_count\_.fetch\_add(1) == 0) \{}@>
  100:       if (pending_head != nullptr /*&& keep_count == pending_count_.load()*/

Local variables: 
pending_head = 0x7fc21bc96070
node = 0x7ffd4ad1d480

<@\textcolor{cyan}{----------------}@>

Object "/home/sham42/git/twist-mc-test/tests/alloc_lfstack3/test.cpp",
in operator()
  61:           SHOW_NOTE("Alloc(): before");
> <@\textcolor{magenta}{62:           stack.Alloc();}@>
  63:           SHOW_NOTE("Alloc(): after");

Local variables: 
available = 0x0
node_index = 0xffffffff

<@\textcolor{cyan}{------------------------------------------------------------------------------------------}@>
                                         <@\textcolor{cyan}{RUNNING}@>
<@\textcolor{cyan}{------------------------------------------------------------------------------------------}@>

fetch_add: prev = 0

<@\textcolor{cyan}{------------------------------------------------------------------------------------------}@>
                                          <@\textcolor{cyan}{AFTER}@>
<@\textcolor{cyan}{------------------------------------------------------------------------------------------}@>

<@\textcolor{red}{SHARED STATE:}@>

    LFAlloc {
	    LFStack {
		    Head {
			    Blocks: -> B (0x7fc21bc96040) -> C (0x7fc21bc96010) 
		    }

		    Pending {
			    Blocks: -> A (0x7fc21bc96070) 
		    }

		<@\textcolor{red}{[*] allocation\_count\_: 1}@>
	    }

	    Memory {
		    C (0x7fc21bc96010) : 1
		    B (0x7fc21bc96040) : 1
		    A (0x7fc21bc96070) : 1
	    }

    }

<@\textcolor{cyan}{------------------------------------------------------------------------------------------}@>
<@\textcolor{magenta}{
STACK:
}@>
Object "/home/sham42/git/twist-mc-test/tests/alloc_lfstack3/alloc_lfstack.hpp",
in LockFreeStack::TryClear(Node*)
  55:     Node* desired = nullptr;
> <@\textcolor{magenta}{56:     return top\_.compare\_exchange\_strong(expected, desired);}@>
  57:   }

Local variables: 
desired = 0x0

<@\textcolor{cyan}{----------------}@>

Object "/home/sham42/git/twist-mc-test/tests/alloc_lfstack3/alloc_lfstack.hpp",
in LFAllocator::Alloc()
  100:       if (pending_head != nullptr /*&& keep_count == pending_count_.load()*/
> <@\textcolor{magenta}{101:           \&\& pending\_list\_.TryClear(pending\_head)) \{}@>
  102:         auto head = pending_head;

Local variables: 
pending_head = 0x7fc21bc96070
node = 0x7ffd4ad1d480

<@\textcolor{cyan}{----------------}@>

Object "/home/sham42/git/twist-mc-test/tests/alloc_lfstack3/test.cpp",
in operator()
  61:           SHOW_NOTE("Alloc(): before");
> <@\textcolor{magenta}{62:           stack.Alloc();}@>
  63:           SHOW_NOTE("Alloc(): after");

Local variables: 
available = 0x0
node_index = 0xffffffff

<@\textcolor{yellow}{==========================================================================================}@>
                                     <@\textcolor{yellow}{RUN 6 (THREAD 3)}@>
<@\textcolor{yellow}{==========================================================================================}@>

<@\textcolor{cyan}{------------------------------------------------------------------------------------------}@>
                                          <@\textcolor{cyan}{BEFORE}@>
<@\textcolor{cyan}{------------------------------------------------------------------------------------------}@>

STATE: STARTING

<@\textcolor{cyan}{------------------------------------------------------------------------------------------}@>
                                         <@\textcolor{cyan}{RUNNING}@>
<@\textcolor{cyan}{------------------------------------------------------------------------------------------}@>

Alloc(): before

<@\textcolor{cyan}{------------------------------------------------------------------------------------------}@>
                                        <@\textcolor{cyan}{SNAPSHOT 1}@>
<@\textcolor{cyan}{------------------------------------------------------------------------------------------}@>

<@\textcolor{red}{SHARED STATE:}@>

    LFAlloc {
	    LFStack {
		    Head {
			    Blocks: -> B (0x7fc21bc96040) -> C (0x7fc21bc96010) 
		    }

		    Pending {
			    Blocks: -> A (0x7fc21bc96070) 
		    }

		    allocation_count_: 1
	    }

	    Memory {
		    C (0x7fc21bc96010) : 1
		    B (0x7fc21bc96040) : 1
		    A (0x7fc21bc96070) : 1
	    }

    }

<@\textcolor{cyan}{------------------------------------------------------------------------------------------}@>
<@\textcolor{magenta}{
STACK:
}@>
Object "/home/sham42/git/twist-mc-test/tests/alloc_lfstack3/alloc_lfstack.hpp",
in LockFreeStack::Top()
  38:   Node* Top() {
> <@\textcolor{magenta}{39:     return top\_.load();}@>
  40:   }

<@\textcolor{cyan}{----------------}@>

Object "/home/sham42/git/twist-mc-test/tests/alloc_lfstack3/alloc_lfstack.hpp",
in LFAllocator::Alloc()
  96:     // auto keep_count = pending_count_.load();
> <@\textcolor{magenta}{97:     Node* pending\_head = pending\_list\_.Top();}@>

Local variables: 
pending_head = 0x7fb80b3b1ea0
node = 0x7ffd4ad1d480

<@\textcolor{cyan}{----------------}@>

Object "/home/sham42/git/twist-mc-test/tests/alloc_lfstack3/test.cpp",
in operator()
  61:           SHOW_NOTE("Alloc(): before");
> <@\textcolor{magenta}{62:           stack.Alloc();}@>
  63:           SHOW_NOTE("Alloc(): after");

Local variables: 
available = 0x0
node_index = 0xffffffff

<@\textcolor{cyan}{------------------------------------------------------------------------------------------}@>
                                         <@\textcolor{cyan}{RUNNING}@>
<@\textcolor{cyan}{------------------------------------------------------------------------------------------}@>

load(): 0x7fc21bc96070

<@\textcolor{cyan}{------------------------------------------------------------------------------------------}@>
                                          <@\textcolor{cyan}{AFTER}@>
<@\textcolor{cyan}{------------------------------------------------------------------------------------------}@>

<@\textcolor{red}{SHARED STATE:}@>

    LFAlloc {
	    LFStack {
		    Head {
			    Blocks: -> B (0x7fc21bc96040) -> C (0x7fc21bc96010) 
		    }

		    Pending {
			    Blocks: -> A (0x7fc21bc96070) 
		    }

		    allocation_count_: 1
	    }

	    Memory {
		    C (0x7fc21bc96010) : 1
		    B (0x7fc21bc96040) : 1
		    A (0x7fc21bc96070) : 1
	    }

    }

<@\textcolor{cyan}{------------------------------------------------------------------------------------------}@>
<@\textcolor{magenta}{
STACK:
}@>
Object "/home/sham42/git/twist-mc-test/tests/alloc_lfstack3/alloc_lfstack.hpp",
in LFAllocator::Alloc()
> <@\textcolor{magenta}{99:     if (allocation\_count\_.fetch\_add(1) == 0) \{}@>
  100:       if (pending_head != nullptr /*&& keep_count == pending_count_.load()*/

Local variables: 
pending_head = 0x7fc21bc96070
node = 0x7ffd4ad1d480

<@\textcolor{cyan}{----------------}@>

Object "/home/sham42/git/twist-mc-test/tests/alloc_lfstack3/test.cpp",
in operator()
  61:           SHOW_NOTE("Alloc(): before");
> <@\textcolor{magenta}{62:           stack.Alloc();}@>
  63:           SHOW_NOTE("Alloc(): after");

Local variables: 
available = 0x0
node_index = 0xffffffff

<@\textcolor{yellow}{==========================================================================================}@>
                                     <@\textcolor{yellow}{RUN 7 (THREAD 2)}@>
<@\textcolor{yellow}{==========================================================================================}@>

<@\textcolor{cyan}{------------------------------------------------------------------------------------------}@>
                                          <@\textcolor{cyan}{BEFORE}@>
<@\textcolor{cyan}{------------------------------------------------------------------------------------------}@>
<@\textcolor{magenta}{
STACK:
}@>
Object "/home/sham42/git/twist-mc-test/tests/alloc_lfstack3/alloc_lfstack.hpp",
in LockFreeStack::TryClear(Node*)
  55:     Node* desired = nullptr;
> <@\textcolor{magenta}{56:     return top\_.compare\_exchange\_strong(expected, desired);}@>
  57:   }

Local variables: 
desired = 0x0

<@\textcolor{cyan}{----------------}@>

Object "/home/sham42/git/twist-mc-test/tests/alloc_lfstack3/alloc_lfstack.hpp",
in LFAllocator::Alloc()
  100:       if (pending_head != nullptr /*&& keep_count == pending_count_.load()*/
> <@\textcolor{magenta}{101:           \&\& pending\_list\_.TryClear(pending\_head)) \{}@>
  102:         auto head = pending_head;

Local variables: 
pending_head = 0x7fc21bc96070
node = 0x7ffd4ad1d480

<@\textcolor{cyan}{----------------}@>

Object "/home/sham42/git/twist-mc-test/tests/alloc_lfstack3/test.cpp",
in operator()
  61:           SHOW_NOTE("Alloc(): before");
> <@\textcolor{magenta}{62:           stack.Alloc();}@>
  63:           SHOW_NOTE("Alloc(): after");

Local variables: 
available = 0x0
node_index = 0xffffffff

<@\textcolor{cyan}{------------------------------------------------------------------------------------------}@>
                                         <@\textcolor{cyan}{RUNNING}@>
<@\textcolor{cyan}{------------------------------------------------------------------------------------------}@>

compare_exchange_strong(): succeeded = 1

<@\textcolor{cyan}{------------------------------------------------------------------------------------------}@>
                                        <@\textcolor{cyan}{SNAPSHOT 1}@>
<@\textcolor{cyan}{------------------------------------------------------------------------------------------}@>

<@\textcolor{red}{SHARED STATE:}@>

    LFAlloc {
	    LFStack {
		    Head {
			    Blocks: -> B (0x7fc21bc96040) -> C (0x7fc21bc96010) 
		    }

		    Pending {
			<@\textcolor{red}{[*] Blocks: }@>
		    }

		    allocation_count_: 1
	    }

	    Memory {
		    C (0x7fc21bc96010) : 1
		    B (0x7fc21bc96040) : 1
		    A (0x7fc21bc96070) : 1
	    }

    }

<@\textcolor{cyan}{------------------------------------------------------------------------------------------}@>
<@\textcolor{magenta}{
STACK:
}@>
Object "/home/sham42/git/twist-mc-test/tests/alloc_lfstack3/alloc_lfstack.hpp",
in LFAllocator::Alloc()
> <@\textcolor{magenta}{104:         pending\_head = pending\_head->next\_.load();}@>
  105:         head->Release();

Local variables: 
head = 0x7fc21bc96070
pending_head = 0x7fc21bc96070
node = 0x7ffd4ad1d480

<@\textcolor{cyan}{----------------}@>

Object "/home/sham42/git/twist-mc-test/tests/alloc_lfstack3/test.cpp",
in operator()
  61:           SHOW_NOTE("Alloc(): before");
> <@\textcolor{magenta}{62:           stack.Alloc();}@>
  63:           SHOW_NOTE("Alloc(): after");

Local variables: 
available = 0x0
node_index = 0xffffffff

<@\textcolor{cyan}{------------------------------------------------------------------------------------------}@>
                                         <@\textcolor{cyan}{RUNNING}@>
<@\textcolor{cyan}{------------------------------------------------------------------------------------------}@>

load(): 0

<@\textcolor{cyan}{------------------------------------------------------------------------------------------}@>
                                        <@\textcolor{cyan}{SNAPSHOT 2}@>
<@\textcolor{cyan}{------------------------------------------------------------------------------------------}@>

<@\textcolor{red}{SHARED STATE:}@>

    LFAlloc {
	    LFStack {
		    Head {
			    Blocks: -> B (0x7fc21bc96040) -> C (0x7fc21bc96010) 
		    }

		    Pending {
			    Blocks: 
		    }

		    allocation_count_: 1
	    }

	    Memory {
		    C (0x7fc21bc96010) : 1
		    B (0x7fc21bc96040) : 1
		<@\textcolor{red}{[*] A (0x7fc21bc96070) : 0}@>
	    }

    }

<@\textcolor{cyan}{------------------------------------------------------------------------------------------}@>
<@\textcolor{magenta}{
STACK:
}@>
Object "/home/sham42/git/twist-mc-test/tests/alloc_lfstack3/alloc_lfstack.hpp",
in LFAllocator::Alloc()
  108:         // pending_count_.fetch_add(1);
> <@\textcolor{magenta}{109:         allocation\_count\_.fetch\_sub(1);}@>

Local variables: 
head = 0x7fc21bc96070
pending_head = 0x0
node = 0x7ffd4ad1d480

<@\textcolor{cyan}{----------------}@>

Object "/home/sham42/git/twist-mc-test/tests/alloc_lfstack3/test.cpp",
in operator()
  61:           SHOW_NOTE("Alloc(): before");
> <@\textcolor{magenta}{62:           stack.Alloc();}@>
  63:           SHOW_NOTE("Alloc(): after");

Local variables: 
available = 0x0
node_index = 0xffffffff

<@\textcolor{cyan}{------------------------------------------------------------------------------------------}@>
                                         <@\textcolor{cyan}{RUNNING}@>
<@\textcolor{cyan}{------------------------------------------------------------------------------------------}@>

fetch_sub(): prev = 1

Alloc(): after

Alloc(): before

<@\textcolor{cyan}{------------------------------------------------------------------------------------------}@>
                                        <@\textcolor{cyan}{SNAPSHOT 3}@>
<@\textcolor{cyan}{------------------------------------------------------------------------------------------}@>

<@\textcolor{red}{SHARED STATE:}@>

    LFAlloc {
	    LFStack {
		    Head {
			    Blocks: -> B (0x7fc21bc96040) -> C (0x7fc21bc96010) 
		    }

		    Pending {
			    Blocks: 
		    }

		<@\textcolor{red}{[*] allocation\_count\_: 0}@>
	    }

	    Memory {
		    C (0x7fc21bc96010) : 1
		    B (0x7fc21bc96040) : 1
		    A (0x7fc21bc96070) : 0
	    }

    }

<@\textcolor{cyan}{------------------------------------------------------------------------------------------}@>
<@\textcolor{magenta}{
STACK:
}@>
Object "/home/sham42/git/twist-mc-test/tests/alloc_lfstack3/alloc_lfstack.hpp",
in LockFreeStack::Top()
  38:   Node* Top() {
> <@\textcolor{magenta}{39:     return top\_.load();}@>
  40:   }

<@\textcolor{cyan}{----------------}@>

Object "/home/sham42/git/twist-mc-test/tests/alloc_lfstack3/alloc_lfstack.hpp",
in LFAllocator::Alloc()
  96:     // auto keep_count = pending_count_.load();
> <@\textcolor{magenta}{97:     Node* pending\_head = pending\_list\_.Top();}@>

Local variables: 
pending_head = 0x7fb80b3baea0
node = 0x7ffd4ad1d480

<@\textcolor{cyan}{----------------}@>

Object "/home/sham42/git/twist-mc-test/tests/alloc_lfstack3/test.cpp",
in operator()
  61:           SHOW_NOTE("Alloc(): before");
> <@\textcolor{magenta}{62:           stack.Alloc();}@>
  63:           SHOW_NOTE("Alloc(): after");

Local variables: 
available = 0x1
node_index = 0xffffffff

<@\textcolor{cyan}{------------------------------------------------------------------------------------------}@>
                                         <@\textcolor{cyan}{RUNNING}@>
<@\textcolor{cyan}{------------------------------------------------------------------------------------------}@>

load(): 0

<@\textcolor{cyan}{------------------------------------------------------------------------------------------}@>
                                          <@\textcolor{cyan}{AFTER}@>
<@\textcolor{cyan}{------------------------------------------------------------------------------------------}@>

<@\textcolor{red}{SHARED STATE:}@>

    LFAlloc {
	    LFStack {
		    Head {
			    Blocks: -> B (0x7fc21bc96040) -> C (0x7fc21bc96010) 
		    }

		    Pending {
			    Blocks: 
		    }

		    allocation_count_: 0
	    }

	    Memory {
		    C (0x7fc21bc96010) : 1
		    B (0x7fc21bc96040) : 1
		    A (0x7fc21bc96070) : 0
	    }

    }

<@\textcolor{cyan}{------------------------------------------------------------------------------------------}@>
<@\textcolor{magenta}{
STACK:
}@>
Object "/home/sham42/git/twist-mc-test/tests/alloc_lfstack3/alloc_lfstack.hpp",
in LFAllocator::Alloc()
> <@\textcolor{magenta}{99:     if (allocation\_count\_.fetch\_add(1) == 0) \{}@>
  100:       if (pending_head != nullptr /*&& keep_count == pending_count_.load()*/

Local variables: 
pending_head = 0x0
node = 0x7ffd4ad1d480

<@\textcolor{cyan}{----------------}@>

Object "/home/sham42/git/twist-mc-test/tests/alloc_lfstack3/test.cpp",
in operator()
  61:           SHOW_NOTE("Alloc(): before");
> <@\textcolor{magenta}{62:           stack.Alloc();}@>
  63:           SHOW_NOTE("Alloc(): after");

Local variables: 
available = 0x1
node_index = 0xffffffff

<@\textcolor{yellow}{==========================================================================================}@>
                                     <@\textcolor{yellow}{RUN 8 (THREAD 3)}@>
<@\textcolor{yellow}{==========================================================================================}@>

<@\textcolor{cyan}{------------------------------------------------------------------------------------------}@>
                                          <@\textcolor{cyan}{BEFORE}@>
<@\textcolor{cyan}{------------------------------------------------------------------------------------------}@>
<@\textcolor{magenta}{
STACK:
}@>
Object "/home/sham42/git/twist-mc-test/tests/alloc_lfstack3/alloc_lfstack.hpp",
in LFAllocator::Alloc()
> <@\textcolor{magenta}{99:     if (allocation\_count\_.fetch\_add(1) == 0) \{}@>
  100:       if (pending_head != nullptr /*&& keep_count == pending_count_.load()*/

Local variables: 
pending_head = 0x7fc21bc96070
node = 0x7ffd4ad1d480

<@\textcolor{cyan}{----------------}@>

Object "/home/sham42/git/twist-mc-test/tests/alloc_lfstack3/test.cpp",
in operator()
  61:           SHOW_NOTE("Alloc(): before");
> <@\textcolor{magenta}{62:           stack.Alloc();}@>
  63:           SHOW_NOTE("Alloc(): after");

Local variables: 
available = 0x0
node_index = 0xffffffff

<@\textcolor{cyan}{------------------------------------------------------------------------------------------}@>
                                         <@\textcolor{cyan}{RUNNING}@>
<@\textcolor{cyan}{------------------------------------------------------------------------------------------}@>

fetch_add: prev = 0

<@\textcolor{cyan}{------------------------------------------------------------------------------------------}@>
                                          <@\textcolor{cyan}{AFTER}@>
<@\textcolor{cyan}{------------------------------------------------------------------------------------------}@>

<@\textcolor{red}{SHARED STATE:}@>

    LFAlloc {
	    LFStack {
		    Head {
			    Blocks: -> B (0x7fc21bc96040) -> C (0x7fc21bc96010) 
		    }

		    Pending {
			    Blocks: 
		    }

		<@\textcolor{red}{[*] allocation\_count\_: 1}@>
	    }

	    Memory {
		    C (0x7fc21bc96010) : 1
		    B (0x7fc21bc96040) : 1
		    A (0x7fc21bc96070) : 0
	    }

    }

<@\textcolor{cyan}{------------------------------------------------------------------------------------------}@>
<@\textcolor{magenta}{
STACK:
}@>
Object "/home/sham42/git/twist-mc-test/tests/alloc_lfstack3/alloc_lfstack.hpp",
in LockFreeStack::TryClear(Node*)
  55:     Node* desired = nullptr;
> <@\textcolor{magenta}{56:     return top\_.compare\_exchange\_strong(expected, desired);}@>
  57:   }

Local variables: 
desired = 0x0

<@\textcolor{cyan}{----------------}@>

Object "/home/sham42/git/twist-mc-test/tests/alloc_lfstack3/alloc_lfstack.hpp",
in LFAllocator::Alloc()
  100:       if (pending_head != nullptr /*&& keep_count == pending_count_.load()*/
> <@\textcolor{magenta}{101:           \&\& pending\_list\_.TryClear(pending\_head)) \{}@>
  102:         auto head = pending_head;

Local variables: 
pending_head = 0x7fc21bc96070
node = 0x7ffd4ad1d480

<@\textcolor{cyan}{----------------}@>

Object "/home/sham42/git/twist-mc-test/tests/alloc_lfstack3/test.cpp",
in operator()
  61:           SHOW_NOTE("Alloc(): before");
> <@\textcolor{magenta}{62:           stack.Alloc();}@>
  63:           SHOW_NOTE("Alloc(): after");

Local variables: 
available = 0x0
node_index = 0xffffffff

<@\textcolor{yellow}{==========================================================================================}@>
                                     <@\textcolor{yellow}{RUN 9 (THREAD 1)}@>
<@\textcolor{yellow}{==========================================================================================}@>

<@\textcolor{cyan}{------------------------------------------------------------------------------------------}@>
                                          <@\textcolor{cyan}{BEFORE}@>
<@\textcolor{cyan}{------------------------------------------------------------------------------------------}@>
<@\textcolor{magenta}{
STACK:
}@>
Object "/home/sham42/git/twist-mc-test/tests/alloc_lfstack3/alloc_lfstack.hpp",
in LFAllocator::Alloc()
> <@\textcolor{magenta}{99:     if (allocation\_count\_.fetch\_add(1) == 0) \{}@>
  100:       if (pending_head != nullptr /*&& keep_count == pending_count_.load()*/

Local variables: 
pending_head = 0x0
node = 0x7ffd4ad1d480

<@\textcolor{cyan}{----------------}@>

Object "/home/sham42/git/twist-mc-test/tests/alloc_lfstack3/test.cpp",
in operator()
  61:           SHOW_NOTE("Alloc(): before");
> <@\textcolor{magenta}{62:           stack.Alloc();}@>
  63:           SHOW_NOTE("Alloc(): after");

Local variables: 
available = 0x0
node_index = 0xffffffff

<@\textcolor{cyan}{------------------------------------------------------------------------------------------}@>
                                         <@\textcolor{cyan}{RUNNING}@>
<@\textcolor{cyan}{------------------------------------------------------------------------------------------}@>

fetch_add: prev = 1

<@\textcolor{cyan}{------------------------------------------------------------------------------------------}@>
                                        <@\textcolor{cyan}{SNAPSHOT 1}@>
<@\textcolor{cyan}{------------------------------------------------------------------------------------------}@>

<@\textcolor{red}{SHARED STATE:}@>

    LFAlloc {
	    LFStack {
		    Head {
			    Blocks: -> B (0x7fc21bc96040) -> C (0x7fc21bc96010) 
		    }

		    Pending {
			    Blocks: 
		    }

		<@\textcolor{red}{[*] allocation\_count\_: 2}@>
	    }

	    Memory {
		    C (0x7fc21bc96010) : 1
		    B (0x7fc21bc96040) : 1
		    A (0x7fc21bc96070) : 0
	    }

    }

<@\textcolor{cyan}{------------------------------------------------------------------------------------------}@>
<@\textcolor{magenta}{
STACK:
}@>
Object "/home/sham42/git/twist-mc-test/tests/alloc_lfstack3/alloc_lfstack.hpp",
in LockFreeStack::Pop()
  42:   Node* Pop() {
> <@\textcolor{magenta}{43:     Node* curr\_top = top\_.load();}@>
  44:     Node* next_top;

Local variables: 
curr_top = 0x100000005
next_top = 0x100000001

<@\textcolor{cyan}{----------------}@>

Object "/home/sham42/git/twist-mc-test/tests/alloc_lfstack3/alloc_lfstack.hpp",
in LFAllocator::Alloc()
> <@\textcolor{magenta}{115:     auto node = item\_list\_.Pop();}@>
  116:     if (node != nullptr) {

Local variables: 
pending_head = 0x0
node = 0x7ffd4ad1d480

<@\textcolor{cyan}{----------------}@>

Object "/home/sham42/git/twist-mc-test/tests/alloc_lfstack3/test.cpp",
in operator()
  61:           SHOW_NOTE("Alloc(): before");
> <@\textcolor{magenta}{62:           stack.Alloc();}@>
  63:           SHOW_NOTE("Alloc(): after");

Local variables: 
available = 0x0
node_index = 0xffffffff

<@\textcolor{cyan}{------------------------------------------------------------------------------------------}@>
                                         <@\textcolor{cyan}{RUNNING}@>
<@\textcolor{cyan}{------------------------------------------------------------------------------------------}@>

load(): 0x7fc21bc96040

<@\textcolor{cyan}{------------------------------------------------------------------------------------------}@>
                                        <@\textcolor{cyan}{SNAPSHOT 2}@>
<@\textcolor{cyan}{------------------------------------------------------------------------------------------}@>

<@\textcolor{red}{SHARED STATE:}@>

    LFAlloc {
	    LFStack {
		    Head {
			    Blocks: -> B (0x7fc21bc96040) -> C (0x7fc21bc96010) 
		    }

		    Pending {
			    Blocks: 
		    }

		    allocation_count_: 2
	    }

	    Memory {
		    C (0x7fc21bc96010) : 1
		    B (0x7fc21bc96040) : 1
		    A (0x7fc21bc96070) : 0
	    }

    }

<@\textcolor{cyan}{------------------------------------------------------------------------------------------}@>
<@\textcolor{magenta}{
STACK:
}@>
Object "/home/sham42/git/twist-mc-test/tests/alloc_lfstack3/alloc_lfstack.hpp",
in LockFreeStack::Pop()
  48:       }
> <@\textcolor{magenta}{49:       next\_top = curr\_top->next\_.load();}@>
  50:     } while (!top_.compare_exchange_weak(curr_top, next_top));

Local variables: 
curr_top = 0x7fc21bc96040
next_top = 0x100000001

<@\textcolor{cyan}{----------------}@>

Object "/home/sham42/git/twist-mc-test/tests/alloc_lfstack3/alloc_lfstack.hpp",
in LFAllocator::Alloc()
> <@\textcolor{magenta}{115:     auto node = item\_list\_.Pop();}@>
  116:     if (node != nullptr) {

Local variables: 
pending_head = 0x0
node = 0x7ffd4ad1d480

<@\textcolor{cyan}{----------------}@>

Object "/home/sham42/git/twist-mc-test/tests/alloc_lfstack3/test.cpp",
in operator()
  61:           SHOW_NOTE("Alloc(): before");
> <@\textcolor{magenta}{62:           stack.Alloc();}@>
  63:           SHOW_NOTE("Alloc(): after");

Local variables: 
available = 0x0
node_index = 0xffffffff

<@\textcolor{cyan}{------------------------------------------------------------------------------------------}@>
                                         <@\textcolor{cyan}{RUNNING}@>
<@\textcolor{cyan}{------------------------------------------------------------------------------------------}@>

load(): 0x7fc21bc96010

<@\textcolor{cyan}{------------------------------------------------------------------------------------------}@>
                                          <@\textcolor{cyan}{AFTER}@>
<@\textcolor{cyan}{------------------------------------------------------------------------------------------}@>

<@\textcolor{red}{SHARED STATE:}@>

    LFAlloc {
	    LFStack {
		    Head {
			    Blocks: -> B (0x7fc21bc96040) -> C (0x7fc21bc96010) 
		    }

		    Pending {
			    Blocks: 
		    }

		    allocation_count_: 2
	    }

	    Memory {
		    C (0x7fc21bc96010) : 1
		    B (0x7fc21bc96040) : 1
		    A (0x7fc21bc96070) : 0
	    }

    }

<@\textcolor{cyan}{------------------------------------------------------------------------------------------}@>
<@\textcolor{magenta}{
STACK:
}@>
Object "/home/sham42/git/twist-mc-test/tests/alloc_lfstack3/alloc_lfstack.hpp",
in LockFreeStack::Pop()
  49:       next_top = curr_top->next_.load();
> <@\textcolor{magenta}{50:     \} while (!top\_.compare\_exchange\_weak(curr\_top, next\_top));}@>
  51:     return curr_top;

Local variables: 
curr_top = 0x7fc21bc96040
next_top = 0x7fc21bc96010

<@\textcolor{cyan}{----------------}@>

Object "/home/sham42/git/twist-mc-test/tests/alloc_lfstack3/alloc_lfstack.hpp",
in LFAllocator::Alloc()
> <@\textcolor{magenta}{115:     auto node = item\_list\_.Pop();}@>
  116:     if (node != nullptr) {

Local variables: 
pending_head = 0x0
node = 0x7ffd4ad1d480

<@\textcolor{cyan}{----------------}@>

Object "/home/sham42/git/twist-mc-test/tests/alloc_lfstack3/test.cpp",
in operator()
  61:           SHOW_NOTE("Alloc(): before");
> <@\textcolor{magenta}{62:           stack.Alloc();}@>
  63:           SHOW_NOTE("Alloc(): after");

Local variables: 
available = 0x0
node_index = 0xffffffff

<@\textcolor{yellow}{==========================================================================================}@>
                                    <@\textcolor{yellow}{RUN 10 (THREAD 2)}@>
<@\textcolor{yellow}{==========================================================================================}@>

<@\textcolor{cyan}{------------------------------------------------------------------------------------------}@>
                                          <@\textcolor{cyan}{BEFORE}@>
<@\textcolor{cyan}{------------------------------------------------------------------------------------------}@>
<@\textcolor{magenta}{
STACK:
}@>
Object "/home/sham42/git/twist-mc-test/tests/alloc_lfstack3/alloc_lfstack.hpp",
in LFAllocator::Alloc()
> <@\textcolor{magenta}{99:     if (allocation\_count\_.fetch\_add(1) == 0) \{}@>
  100:       if (pending_head != nullptr /*&& keep_count == pending_count_.load()*/

Local variables: 
pending_head = 0x0
node = 0x7ffd4ad1d480

<@\textcolor{cyan}{----------------}@>

Object "/home/sham42/git/twist-mc-test/tests/alloc_lfstack3/test.cpp",
in operator()
  61:           SHOW_NOTE("Alloc(): before");
> <@\textcolor{magenta}{62:           stack.Alloc();}@>
  63:           SHOW_NOTE("Alloc(): after");

Local variables: 
available = 0x1
node_index = 0xffffffff

<@\textcolor{cyan}{------------------------------------------------------------------------------------------}@>
                                         <@\textcolor{cyan}{RUNNING}@>
<@\textcolor{cyan}{------------------------------------------------------------------------------------------}@>

fetch_add: prev = 2

<@\textcolor{cyan}{------------------------------------------------------------------------------------------}@>
                                        <@\textcolor{cyan}{SNAPSHOT 1}@>
<@\textcolor{cyan}{------------------------------------------------------------------------------------------}@>

<@\textcolor{red}{SHARED STATE:}@>

    LFAlloc {
	    LFStack {
		    Head {
			    Blocks: -> B (0x7fc21bc96040) -> C (0x7fc21bc96010) 
		    }

		    Pending {
			    Blocks: 
		    }

		<@\textcolor{red}{[*] allocation\_count\_: 3}@>
	    }

	    Memory {
		    C (0x7fc21bc96010) : 1
		    B (0x7fc21bc96040) : 1
		    A (0x7fc21bc96070) : 0
	    }

    }

<@\textcolor{cyan}{------------------------------------------------------------------------------------------}@>
<@\textcolor{magenta}{
STACK:
}@>
Object "/home/sham42/git/twist-mc-test/tests/alloc_lfstack3/alloc_lfstack.hpp",
in LockFreeStack::Pop()
  42:   Node* Pop() {
> <@\textcolor{magenta}{43:     Node* curr\_top = top\_.load();}@>
  44:     Node* next_top;

Local variables: 
curr_top = 0x100000005
next_top = 0x100000002

<@\textcolor{cyan}{----------------}@>

Object "/home/sham42/git/twist-mc-test/tests/alloc_lfstack3/alloc_lfstack.hpp",
in LFAllocator::Alloc()
> <@\textcolor{magenta}{115:     auto node = item\_list\_.Pop();}@>
  116:     if (node != nullptr) {

Local variables: 
pending_head = 0x0
node = 0x7ffd4ad1d480

<@\textcolor{cyan}{----------------}@>

Object "/home/sham42/git/twist-mc-test/tests/alloc_lfstack3/test.cpp",
in operator()
  61:           SHOW_NOTE("Alloc(): before");
> <@\textcolor{magenta}{62:           stack.Alloc();}@>
  63:           SHOW_NOTE("Alloc(): after");

Local variables: 
available = 0x1
node_index = 0xffffffff

<@\textcolor{cyan}{------------------------------------------------------------------------------------------}@>
                                         <@\textcolor{cyan}{RUNNING}@>
<@\textcolor{cyan}{------------------------------------------------------------------------------------------}@>

load(): 0x7fc21bc96040

<@\textcolor{cyan}{------------------------------------------------------------------------------------------}@>
                                          <@\textcolor{cyan}{AFTER}@>
<@\textcolor{cyan}{------------------------------------------------------------------------------------------}@>

<@\textcolor{red}{SHARED STATE:}@>

    LFAlloc {
	    LFStack {
		    Head {
			    Blocks: -> B (0x7fc21bc96040) -> C (0x7fc21bc96010) 
		    }

		    Pending {
			    Blocks: 
		    }

		    allocation_count_: 3
	    }

	    Memory {
		    C (0x7fc21bc96010) : 1
		    B (0x7fc21bc96040) : 1
		    A (0x7fc21bc96070) : 0
	    }

    }

<@\textcolor{cyan}{------------------------------------------------------------------------------------------}@>
<@\textcolor{magenta}{
STACK:
}@>
Object "/home/sham42/git/twist-mc-test/tests/alloc_lfstack3/alloc_lfstack.hpp",
in LockFreeStack::Pop()
  48:       }
> <@\textcolor{magenta}{49:       next\_top = curr\_top->next\_.load();}@>
  50:     } while (!top_.compare_exchange_weak(curr_top, next_top));

Local variables: 
curr_top = 0x7fc21bc96040
next_top = 0x100000002

<@\textcolor{cyan}{----------------}@>

Object "/home/sham42/git/twist-mc-test/tests/alloc_lfstack3/alloc_lfstack.hpp",
in LFAllocator::Alloc()
> <@\textcolor{magenta}{115:     auto node = item\_list\_.Pop();}@>
  116:     if (node != nullptr) {

Local variables: 
pending_head = 0x0
node = 0x7ffd4ad1d480

<@\textcolor{cyan}{----------------}@>

Object "/home/sham42/git/twist-mc-test/tests/alloc_lfstack3/test.cpp",
in operator()
  61:           SHOW_NOTE("Alloc(): before");
> <@\textcolor{magenta}{62:           stack.Alloc();}@>
  63:           SHOW_NOTE("Alloc(): after");

Local variables: 
available = 0x1
node_index = 0xffffffff

<@\textcolor{yellow}{==========================================================================================}@>
                                    <@\textcolor{yellow}{RUN 11 (THREAD 1)}@>
<@\textcolor{yellow}{==========================================================================================}@>

<@\textcolor{cyan}{------------------------------------------------------------------------------------------}@>
                                          <@\textcolor{cyan}{BEFORE}@>
<@\textcolor{cyan}{------------------------------------------------------------------------------------------}@>
<@\textcolor{magenta}{
STACK:
}@>
Object "/home/sham42/git/twist-mc-test/tests/alloc_lfstack3/alloc_lfstack.hpp",
in LockFreeStack::Pop()
  49:       next_top = curr_top->next_.load();
> <@\textcolor{magenta}{50:     \} while (!top\_.compare\_exchange\_weak(curr\_top, next\_top));}@>
  51:     return curr_top;

Local variables: 
curr_top = 0x7fc21bc96040
next_top = 0x7fc21bc96010

<@\textcolor{cyan}{----------------}@>

Object "/home/sham42/git/twist-mc-test/tests/alloc_lfstack3/alloc_lfstack.hpp",
in LFAllocator::Alloc()
> <@\textcolor{magenta}{115:     auto node = item\_list\_.Pop();}@>
  116:     if (node != nullptr) {

Local variables: 
pending_head = 0x0
node = 0x7ffd4ad1d480

<@\textcolor{cyan}{----------------}@>

Object "/home/sham42/git/twist-mc-test/tests/alloc_lfstack3/test.cpp",
in operator()
  61:           SHOW_NOTE("Alloc(): before");
> <@\textcolor{magenta}{62:           stack.Alloc();}@>
  63:           SHOW_NOTE("Alloc(): after");

Local variables: 
available = 0x0
node_index = 0xffffffff

<@\textcolor{cyan}{------------------------------------------------------------------------------------------}@>
                                         <@\textcolor{cyan}{RUNNING}@>
<@\textcolor{cyan}{------------------------------------------------------------------------------------------}@>

compare_exchange_weak(): succeeded = 1

<@\textcolor{cyan}{------------------------------------------------------------------------------------------}@>
                                        <@\textcolor{cyan}{SNAPSHOT 1}@>
<@\textcolor{cyan}{------------------------------------------------------------------------------------------}@>

<@\textcolor{red}{SHARED STATE:}@>

    LFAlloc {
	    LFStack {
		    Head {
			<@\textcolor{red}{[*] Blocks: -> C (0x7fc21bc96010) }@>
		    }

		    Pending {
			    Blocks: 
		    }

		    allocation_count_: 3
	    }

	    Memory {
		    C (0x7fc21bc96010) : 1
		<@\textcolor{red}{[*] B (0x7fc21bc96040) : 0}@>
		    A (0x7fc21bc96070) : 0
	    }

    }

<@\textcolor{cyan}{------------------------------------------------------------------------------------------}@>
<@\textcolor{magenta}{
STACK:
}@>
Object "/home/sham42/git/twist-mc-test/tests/alloc_lfstack3/alloc_lfstack.hpp",
in LFAllocator::Alloc()
> <@\textcolor{magenta}{120:     allocation\_count\_.fetch\_sub(1);}@>

Local variables: 
pending_head = 0x0
node = 0x7fc21bc96040

<@\textcolor{cyan}{----------------}@>

Object "/home/sham42/git/twist-mc-test/tests/alloc_lfstack3/test.cpp",
in operator()
  61:           SHOW_NOTE("Alloc(): before");
> <@\textcolor{magenta}{62:           stack.Alloc();}@>
  63:           SHOW_NOTE("Alloc(): after");

Local variables: 
available = 0x0
node_index = 0xffffffff

<@\textcolor{cyan}{------------------------------------------------------------------------------------------}@>
                                         <@\textcolor{cyan}{RUNNING}@>
<@\textcolor{cyan}{------------------------------------------------------------------------------------------}@>

fetch_sub(): prev = 3

Alloc(): after

Free(): before

<@\textcolor{cyan}{------------------------------------------------------------------------------------------}@>
                                        <@\textcolor{cyan}{SNAPSHOT 2}@>
<@\textcolor{cyan}{------------------------------------------------------------------------------------------}@>

<@\textcolor{red}{SHARED STATE:}@>

    LFAlloc {
	    LFStack {
		    Head {
			    Blocks: -> C (0x7fc21bc96010) 
		    }

		    Pending {
			    Blocks: 
		    }

		<@\textcolor{red}{[*] allocation\_count\_: 2}@>
	    }

	    Memory {
		    C (0x7fc21bc96010) : 1
		<@\textcolor{red}{[*] B (0x7fc21bc96040) : 1}@>
		    A (0x7fc21bc96070) : 0
	    }

    }

<@\textcolor{cyan}{------------------------------------------------------------------------------------------}@>
<@\textcolor{magenta}{
STACK:
}@>
Object "/home/sham42/git/twist-mc-test/tests/alloc_lfstack3/alloc_lfstack.hpp",
in LFAllocator::Free(Node*)
  87:   void Free(Node* node) /*__attribute__((noinline))*/ {
> <@\textcolor{magenta}{88:     if (allocation\_count\_.load() == 0) \{}@>
  89:       item_list_.Push(node);

<@\textcolor{cyan}{----------------}@>

Object "/home/sham42/git/twist-mc-test/tests/alloc_lfstack3/test.cpp",
in operator()
  66:           Node* to_free = pool.Acquire(node_index);
> <@\textcolor{magenta}{67:           stack.Free(to\_free);}@>
  68:           SHOW_NOTE("Free(): after");

Local variables: 
to_free = 0x7fc21bc96040
available = 0x2
node_index = 0x0

<@\textcolor{cyan}{------------------------------------------------------------------------------------------}@>
                                         <@\textcolor{cyan}{RUNNING}@>
<@\textcolor{cyan}{------------------------------------------------------------------------------------------}@>

load(): 2

<@\textcolor{cyan}{------------------------------------------------------------------------------------------}@>
                                        <@\textcolor{cyan}{SNAPSHOT 3}@>
<@\textcolor{cyan}{------------------------------------------------------------------------------------------}@>

<@\textcolor{red}{SHARED STATE:}@>

    LFAlloc {
	    LFStack {
		    Head {
			    Blocks: -> C (0x7fc21bc96010) 
		    }

		    Pending {
			    Blocks: 
		    }

		    allocation_count_: 2
	    }

	    Memory {
		    C (0x7fc21bc96010) : 1
		    B (0x7fc21bc96040) : 1
		    A (0x7fc21bc96070) : 0
	    }

    }

<@\textcolor{cyan}{------------------------------------------------------------------------------------------}@>
<@\textcolor{magenta}{
STACK:
}@>
Object "/home/sham42/git/twist-mc-test/tests/alloc_lfstack3/alloc_lfstack.hpp",
in LockFreeStack::Push(Node*)
  13:   void Push(Node* node) {
> <@\textcolor{magenta}{14:     Node* current\_top = top\_.load();}@>
  15:     do {

Local variables: 
current_top = 0x200413133

<@\textcolor{cyan}{----------------}@>

Object "/home/sham42/git/twist-mc-test/tests/alloc_lfstack3/alloc_lfstack.hpp",
in LFAllocator::Free(Node*)
  90:     } else {
> <@\textcolor{magenta}{91:       pending\_list\_.Push(node);}@>
  92:     }

<@\textcolor{cyan}{----------------}@>

Object "/home/sham42/git/twist-mc-test/tests/alloc_lfstack3/test.cpp",
in operator()
  66:           Node* to_free = pool.Acquire(node_index);
> <@\textcolor{magenta}{67:           stack.Free(to\_free);}@>
  68:           SHOW_NOTE("Free(): after");

Local variables: 
to_free = 0x7fc21bc96040
available = 0x2
node_index = 0x0

<@\textcolor{cyan}{------------------------------------------------------------------------------------------}@>
                                         <@\textcolor{cyan}{RUNNING}@>
<@\textcolor{cyan}{------------------------------------------------------------------------------------------}@>

load(): 0

<@\textcolor{cyan}{------------------------------------------------------------------------------------------}@>
                                          <@\textcolor{cyan}{AFTER}@>
<@\textcolor{cyan}{------------------------------------------------------------------------------------------}@>

<@\textcolor{red}{SHARED STATE:}@>

    LFAlloc {
	    LFStack {
		    Head {
			    Blocks: -> C (0x7fc21bc96010) 
		    }

		    Pending {
			    Blocks: 
		    }

		    allocation_count_: 2
	    }

	    Memory {
		    C (0x7fc21bc96010) : 1
		    B (0x7fc21bc96040) : 1
		    A (0x7fc21bc96070) : 0
	    }

    }

<@\textcolor{cyan}{------------------------------------------------------------------------------------------}@>
<@\textcolor{magenta}{
STACK:
}@>
Object "/home/sham42/git/twist-mc-test/tests/alloc_lfstack3/alloc_lfstack.hpp",
in LockFreeStack::Push(Node*)
  15:     do {
> <@\textcolor{magenta}{16:       node->next\_.store(current\_top);}@>
  17:     } while (!top_.compare_exchange_weak(current_top, node));

Local variables: 
current_top = 0x0

<@\textcolor{cyan}{----------------}@>

Object "/home/sham42/git/twist-mc-test/tests/alloc_lfstack3/alloc_lfstack.hpp",
in LFAllocator::Free(Node*)
  90:     } else {
> <@\textcolor{magenta}{91:       pending\_list\_.Push(node);}@>
  92:     }

<@\textcolor{cyan}{----------------}@>

Object "/home/sham42/git/twist-mc-test/tests/alloc_lfstack3/test.cpp",
in operator()
  66:           Node* to_free = pool.Acquire(node_index);
> <@\textcolor{magenta}{67:           stack.Free(to\_free);}@>
  68:           SHOW_NOTE("Free(): after");

Local variables: 
to_free = 0x7fc21bc96040
available = 0x2
node_index = 0x0

<@\textcolor{yellow}{==========================================================================================}@>
                                    <@\textcolor{yellow}{RUN 12 (THREAD 2)}@>
<@\textcolor{yellow}{==========================================================================================}@>

<@\textcolor{cyan}{------------------------------------------------------------------------------------------}@>
                                          <@\textcolor{cyan}{BEFORE}@>
<@\textcolor{cyan}{------------------------------------------------------------------------------------------}@>
<@\textcolor{magenta}{
STACK:
}@>
Object "/home/sham42/git/twist-mc-test/tests/alloc_lfstack3/alloc_lfstack.hpp",
in LockFreeStack::Pop()
  48:       }
> <@\textcolor{magenta}{49:       next\_top = curr\_top->next\_.load();}@>
  50:     } while (!top_.compare_exchange_weak(curr_top, next_top));

Local variables: 
curr_top = 0x7fc21bc96040
next_top = 0x100000002

<@\textcolor{cyan}{----------------}@>

Object "/home/sham42/git/twist-mc-test/tests/alloc_lfstack3/alloc_lfstack.hpp",
in LFAllocator::Alloc()
> <@\textcolor{magenta}{115:     auto node = item\_list\_.Pop();}@>
  116:     if (node != nullptr) {

Local variables: 
pending_head = 0x0
node = 0x7ffd4ad1d480

<@\textcolor{cyan}{----------------}@>

Object "/home/sham42/git/twist-mc-test/tests/alloc_lfstack3/test.cpp",
in operator()
  61:           SHOW_NOTE("Alloc(): before");
> <@\textcolor{magenta}{62:           stack.Alloc();}@>
  63:           SHOW_NOTE("Alloc(): after");

Local variables: 
available = 0x1
node_index = 0xffffffff

<@\textcolor{cyan}{------------------------------------------------------------------------------------------}@>
                                         <@\textcolor{cyan}{RUNNING}@>
<@\textcolor{cyan}{------------------------------------------------------------------------------------------}@>

load(): 0x7fc21bc96010

<@\textcolor{cyan}{------------------------------------------------------------------------------------------}@>
                                          <@\textcolor{cyan}{AFTER}@>
<@\textcolor{cyan}{------------------------------------------------------------------------------------------}@>

<@\textcolor{red}{SHARED STATE:}@>

    LFAlloc {
	    LFStack {
		    Head {
			    Blocks: -> C (0x7fc21bc96010) 
		    }

		    Pending {
			    Blocks: 
		    }

		    allocation_count_: 2
	    }

	    Memory {
		    C (0x7fc21bc96010) : 1
		    B (0x7fc21bc96040) : 1
		    A (0x7fc21bc96070) : 0
	    }

    }

<@\textcolor{cyan}{------------------------------------------------------------------------------------------}@>
<@\textcolor{magenta}{
STACK:
}@>
Object "/home/sham42/git/twist-mc-test/tests/alloc_lfstack3/alloc_lfstack.hpp",
in LockFreeStack::Pop()
  49:       next_top = curr_top->next_.load();
> <@\textcolor{magenta}{50:     \} while (!top\_.compare\_exchange\_weak(curr\_top, next\_top));}@>
  51:     return curr_top;

Local variables: 
curr_top = 0x7fc21bc96040
next_top = 0x7fc21bc96010

<@\textcolor{cyan}{----------------}@>

Object "/home/sham42/git/twist-mc-test/tests/alloc_lfstack3/alloc_lfstack.hpp",
in LFAllocator::Alloc()
> <@\textcolor{magenta}{115:     auto node = item\_list\_.Pop();}@>
  116:     if (node != nullptr) {

Local variables: 
pending_head = 0x0
node = 0x7ffd4ad1d480

<@\textcolor{cyan}{----------------}@>

Object "/home/sham42/git/twist-mc-test/tests/alloc_lfstack3/test.cpp",
in operator()
  61:           SHOW_NOTE("Alloc(): before");
> <@\textcolor{magenta}{62:           stack.Alloc();}@>
  63:           SHOW_NOTE("Alloc(): after");

Local variables: 
available = 0x1
node_index = 0xffffffff

<@\textcolor{yellow}{==========================================================================================}@>
                                    <@\textcolor{yellow}{RUN 13 (THREAD 1)}@>
<@\textcolor{yellow}{==========================================================================================}@>

<@\textcolor{cyan}{------------------------------------------------------------------------------------------}@>
                                          <@\textcolor{cyan}{BEFORE}@>
<@\textcolor{cyan}{------------------------------------------------------------------------------------------}@>
<@\textcolor{magenta}{
STACK:
}@>
Object "/home/sham42/git/twist-mc-test/tests/alloc_lfstack3/alloc_lfstack.hpp",
in LockFreeStack::Push(Node*)
  15:     do {
> <@\textcolor{magenta}{16:       node->next\_.store(current\_top);}@>
  17:     } while (!top_.compare_exchange_weak(current_top, node));

Local variables: 
current_top = 0x0

<@\textcolor{cyan}{----------------}@>

Object "/home/sham42/git/twist-mc-test/tests/alloc_lfstack3/alloc_lfstack.hpp",
in LFAllocator::Free(Node*)
  90:     } else {
> <@\textcolor{magenta}{91:       pending\_list\_.Push(node);}@>
  92:     }

<@\textcolor{cyan}{----------------}@>

Object "/home/sham42/git/twist-mc-test/tests/alloc_lfstack3/test.cpp",
in operator()
  66:           Node* to_free = pool.Acquire(node_index);
> <@\textcolor{magenta}{67:           stack.Free(to\_free);}@>
  68:           SHOW_NOTE("Free(): after");

Local variables: 
to_free = 0x7fc21bc96040
available = 0x2
node_index = 0x0

<@\textcolor{cyan}{------------------------------------------------------------------------------------------}@>
                                        <@\textcolor{cyan}{SNAPSHOT 1}@>
<@\textcolor{cyan}{------------------------------------------------------------------------------------------}@>

<@\textcolor{red}{SHARED STATE:}@>

    LFAlloc {
	    LFStack {
		    Head {
			    Blocks: -> C (0x7fc21bc96010) 
		    }

		    Pending {
			    Blocks: 
		    }

		    allocation_count_: 2
	    }

	    Memory {
		    C (0x7fc21bc96010) : 1
		    B (0x7fc21bc96040) : 1
		    A (0x7fc21bc96070) : 0
	    }

    }

<@\textcolor{cyan}{------------------------------------------------------------------------------------------}@>
<@\textcolor{magenta}{
STACK:
}@>
Object "/home/sham42/git/twist-mc-test/tests/alloc_lfstack3/alloc_lfstack.hpp",
in LockFreeStack::Push(Node*)
  16:       node->next_.store(current_top);
> <@\textcolor{magenta}{17:     \} while (!top\_.compare\_exchange\_weak(current\_top, node));}@>
  18:   }

Local variables: 
current_top = 0x0

<@\textcolor{cyan}{----------------}@>

Object "/home/sham42/git/twist-mc-test/tests/alloc_lfstack3/alloc_lfstack.hpp",
in LFAllocator::Free(Node*)
  90:     } else {
> <@\textcolor{magenta}{91:       pending\_list\_.Push(node);}@>
  92:     }

<@\textcolor{cyan}{----------------}@>

Object "/home/sham42/git/twist-mc-test/tests/alloc_lfstack3/test.cpp",
in operator()
  66:           Node* to_free = pool.Acquire(node_index);
> <@\textcolor{magenta}{67:           stack.Free(to\_free);}@>
  68:           SHOW_NOTE("Free(): after");

Local variables: 
to_free = 0x7fc21bc96040
available = 0x2
node_index = 0x0

<@\textcolor{cyan}{------------------------------------------------------------------------------------------}@>
                                         <@\textcolor{cyan}{RUNNING}@>
<@\textcolor{cyan}{------------------------------------------------------------------------------------------}@>

compare_exchange_weak(): succeeded = 1

Free(): after

Free(): before

<@\textcolor{cyan}{------------------------------------------------------------------------------------------}@>
                                        <@\textcolor{cyan}{SNAPSHOT 2}@>
<@\textcolor{cyan}{------------------------------------------------------------------------------------------}@>

<@\textcolor{red}{SHARED STATE:}@>

    LFAlloc {
	    LFStack {
		    Head {
			    Blocks: -> C (0x7fc21bc96010) 
		    }

		    Pending {
			<@\textcolor{red}{[*] Blocks: -> B (0x7fc21bc96040) }@>
		    }

		    allocation_count_: 2
	    }

	    Memory {
		    C (0x7fc21bc96010) : 1
		    B (0x7fc21bc96040) : 1
		<@\textcolor{red}{[*] A (0x7fc21bc96070) : 1}@>
	    }

    }

<@\textcolor{cyan}{------------------------------------------------------------------------------------------}@>
<@\textcolor{magenta}{
STACK:
}@>
Object "/home/sham42/git/twist-mc-test/tests/alloc_lfstack3/alloc_lfstack.hpp",
in LFAllocator::Free(Node*)
  87:   void Free(Node* node) /*__attribute__((noinline))*/ {
> <@\textcolor{magenta}{88:     if (allocation\_count\_.load() == 0) \{}@>
  89:       item_list_.Push(node);

<@\textcolor{cyan}{----------------}@>

Object "/home/sham42/git/twist-mc-test/tests/alloc_lfstack3/test.cpp",
in operator()
  66:           Node* to_free = pool.Acquire(node_index);
> <@\textcolor{magenta}{67:           stack.Free(to\_free);}@>
  68:           SHOW_NOTE("Free(): after");

Local variables: 
to_free = 0x7fc21bc96070
available = 0x1
node_index = 0x0

<@\textcolor{cyan}{------------------------------------------------------------------------------------------}@>
                                         <@\textcolor{cyan}{RUNNING}@>
<@\textcolor{cyan}{------------------------------------------------------------------------------------------}@>

load(): 2

<@\textcolor{cyan}{------------------------------------------------------------------------------------------}@>
                                        <@\textcolor{cyan}{SNAPSHOT 3}@>
<@\textcolor{cyan}{------------------------------------------------------------------------------------------}@>

<@\textcolor{red}{SHARED STATE:}@>

    LFAlloc {
	    LFStack {
		    Head {
			    Blocks: -> C (0x7fc21bc96010) 
		    }

		    Pending {
			    Blocks: -> B (0x7fc21bc96040) 
		    }

		    allocation_count_: 2
	    }

	    Memory {
		    C (0x7fc21bc96010) : 1
		    B (0x7fc21bc96040) : 1
		    A (0x7fc21bc96070) : 1
	    }

    }

<@\textcolor{cyan}{------------------------------------------------------------------------------------------}@>
<@\textcolor{magenta}{
STACK:
}@>
Object "/home/sham42/git/twist-mc-test/tests/alloc_lfstack3/alloc_lfstack.hpp",
in LockFreeStack::Push(Node*)
  13:   void Push(Node* node) {
> <@\textcolor{magenta}{14:     Node* current\_top = top\_.load();}@>
  15:     do {

Local variables: 
current_top = 0x200413133

<@\textcolor{cyan}{----------------}@>

Object "/home/sham42/git/twist-mc-test/tests/alloc_lfstack3/alloc_lfstack.hpp",
in LFAllocator::Free(Node*)
  90:     } else {
> <@\textcolor{magenta}{91:       pending\_list\_.Push(node);}@>
  92:     }

<@\textcolor{cyan}{----------------}@>

Object "/home/sham42/git/twist-mc-test/tests/alloc_lfstack3/test.cpp",
in operator()
  66:           Node* to_free = pool.Acquire(node_index);
> <@\textcolor{magenta}{67:           stack.Free(to\_free);}@>
  68:           SHOW_NOTE("Free(): after");

Local variables: 
to_free = 0x7fc21bc96070
available = 0x1
node_index = 0x0

<@\textcolor{cyan}{------------------------------------------------------------------------------------------}@>
                                         <@\textcolor{cyan}{RUNNING}@>
<@\textcolor{cyan}{------------------------------------------------------------------------------------------}@>

load(): 0x7fc21bc96040

<@\textcolor{cyan}{------------------------------------------------------------------------------------------}@>
                                        <@\textcolor{cyan}{SNAPSHOT 4}@>
<@\textcolor{cyan}{------------------------------------------------------------------------------------------}@>

<@\textcolor{red}{SHARED STATE:}@>

    LFAlloc {
	    LFStack {
		    Head {
			    Blocks: -> C (0x7fc21bc96010) 
		    }

		    Pending {
			    Blocks: -> B (0x7fc21bc96040) 
		    }

		    allocation_count_: 2
	    }

	    Memory {
		    C (0x7fc21bc96010) : 1
		    B (0x7fc21bc96040) : 1
		    A (0x7fc21bc96070) : 1
	    }

    }

<@\textcolor{cyan}{------------------------------------------------------------------------------------------}@>
<@\textcolor{magenta}{
STACK:
}@>
Object "/home/sham42/git/twist-mc-test/tests/alloc_lfstack3/alloc_lfstack.hpp",
in LockFreeStack::Push(Node*)
  15:     do {
> <@\textcolor{magenta}{16:       node->next\_.store(current\_top);}@>
  17:     } while (!top_.compare_exchange_weak(current_top, node));

Local variables: 
current_top = 0x7fc21bc96040

<@\textcolor{cyan}{----------------}@>

Object "/home/sham42/git/twist-mc-test/tests/alloc_lfstack3/alloc_lfstack.hpp",
in LFAllocator::Free(Node*)
  90:     } else {
> <@\textcolor{magenta}{91:       pending\_list\_.Push(node);}@>
  92:     }

<@\textcolor{cyan}{----------------}@>

Object "/home/sham42/git/twist-mc-test/tests/alloc_lfstack3/test.cpp",
in operator()
  66:           Node* to_free = pool.Acquire(node_index);
> <@\textcolor{magenta}{67:           stack.Free(to\_free);}@>
  68:           SHOW_NOTE("Free(): after");

Local variables: 
to_free = 0x7fc21bc96070
available = 0x1
node_index = 0x0

<@\textcolor{cyan}{------------------------------------------------------------------------------------------}@>
                                        <@\textcolor{cyan}{SNAPSHOT 5}@>
<@\textcolor{cyan}{------------------------------------------------------------------------------------------}@>

<@\textcolor{red}{SHARED STATE:}@>

    LFAlloc {
	    LFStack {
		    Head {
			    Blocks: -> C (0x7fc21bc96010) 
		    }

		    Pending {
			    Blocks: -> B (0x7fc21bc96040) 
		    }

		    allocation_count_: 2
	    }

	    Memory {
		    C (0x7fc21bc96010) : 1
		    B (0x7fc21bc96040) : 1
		    A (0x7fc21bc96070) : 1
	    }

    }

<@\textcolor{cyan}{------------------------------------------------------------------------------------------}@>
<@\textcolor{magenta}{
STACK:
}@>
Object "/home/sham42/git/twist-mc-test/tests/alloc_lfstack3/alloc_lfstack.hpp",
in LockFreeStack::Push(Node*)
  16:       node->next_.store(current_top);
> <@\textcolor{magenta}{17:     \} while (!top\_.compare\_exchange\_weak(current\_top, node));}@>
  18:   }

Local variables: 
current_top = 0x7fc21bc96040

<@\textcolor{cyan}{----------------}@>

Object "/home/sham42/git/twist-mc-test/tests/alloc_lfstack3/alloc_lfstack.hpp",
in LFAllocator::Free(Node*)
  90:     } else {
> <@\textcolor{magenta}{91:       pending\_list\_.Push(node);}@>
  92:     }

<@\textcolor{cyan}{----------------}@>

Object "/home/sham42/git/twist-mc-test/tests/alloc_lfstack3/test.cpp",
in operator()
  66:           Node* to_free = pool.Acquire(node_index);
> <@\textcolor{magenta}{67:           stack.Free(to\_free);}@>
  68:           SHOW_NOTE("Free(): after");

Local variables: 
to_free = 0x7fc21bc96070
available = 0x1
node_index = 0x0

<@\textcolor{cyan}{------------------------------------------------------------------------------------------}@>
                                         <@\textcolor{cyan}{RUNNING}@>
<@\textcolor{cyan}{------------------------------------------------------------------------------------------}@>

compare_exchange_weak(): succeeded = 1

Free(): after

Alloc(): before

<@\textcolor{cyan}{------------------------------------------------------------------------------------------}@>
                                        <@\textcolor{cyan}{SNAPSHOT 6}@>
<@\textcolor{cyan}{------------------------------------------------------------------------------------------}@>

<@\textcolor{red}{SHARED STATE:}@>

    LFAlloc {
	    LFStack {
		    Head {
			    Blocks: -> C (0x7fc21bc96010) 
		    }

		    Pending {
			<@\textcolor{red}{[*] Blocks: -> A (0x7fc21bc96070) -> B (0x7fc21bc96040) }@>
		    }

		    allocation_count_: 2
	    }

	    Memory {
		    C (0x7fc21bc96010) : 1
		    B (0x7fc21bc96040) : 1
		    A (0x7fc21bc96070) : 1
	    }

    }

<@\textcolor{cyan}{------------------------------------------------------------------------------------------}@>
<@\textcolor{magenta}{
STACK:
}@>
Object "/home/sham42/git/twist-mc-test/tests/alloc_lfstack3/alloc_lfstack.hpp",
in LockFreeStack::Top()
  38:   Node* Top() {
> <@\textcolor{magenta}{39:     return top\_.load();}@>
  40:   }

<@\textcolor{cyan}{----------------}@>

Object "/home/sham42/git/twist-mc-test/tests/alloc_lfstack3/alloc_lfstack.hpp",
in LFAllocator::Alloc()
  96:     // auto keep_count = pending_count_.load();
> <@\textcolor{magenta}{97:     Node* pending\_head = pending\_list\_.Top();}@>

Local variables: 
pending_head = 0x7fc21bc8cea0
node = 0x7ffd4ad1d480

<@\textcolor{cyan}{----------------}@>

Object "/home/sham42/git/twist-mc-test/tests/alloc_lfstack3/test.cpp",
in operator()
  61:           SHOW_NOTE("Alloc(): before");
> <@\textcolor{magenta}{62:           stack.Alloc();}@>
  63:           SHOW_NOTE("Alloc(): after");

Local variables: 
available = 0x0
node_index = 0xffffffff

<@\textcolor{cyan}{------------------------------------------------------------------------------------------}@>
                                         <@\textcolor{cyan}{RUNNING}@>
<@\textcolor{cyan}{------------------------------------------------------------------------------------------}@>

load(): 0x7fc21bc96070

<@\textcolor{cyan}{------------------------------------------------------------------------------------------}@>
                                        <@\textcolor{cyan}{SNAPSHOT 7}@>
<@\textcolor{cyan}{------------------------------------------------------------------------------------------}@>

<@\textcolor{red}{SHARED STATE:}@>

    LFAlloc {
	    LFStack {
		    Head {
			    Blocks: -> C (0x7fc21bc96010) 
		    }

		    Pending {
			    Blocks: -> A (0x7fc21bc96070) -> B (0x7fc21bc96040) 
		    }

		    allocation_count_: 2
	    }

	    Memory {
		    C (0x7fc21bc96010) : 1
		    B (0x7fc21bc96040) : 1
		    A (0x7fc21bc96070) : 1
	    }

    }

<@\textcolor{cyan}{------------------------------------------------------------------------------------------}@>
<@\textcolor{magenta}{
STACK:
}@>
Object "/home/sham42/git/twist-mc-test/tests/alloc_lfstack3/alloc_lfstack.hpp",
in LFAllocator::Alloc()
> <@\textcolor{magenta}{99:     if (allocation\_count\_.fetch\_add(1) == 0) \{}@>
  100:       if (pending_head != nullptr /*&& keep_count == pending_count_.load()*/

Local variables: 
pending_head = 0x7fc21bc96070
node = 0x7ffd4ad1d480

<@\textcolor{cyan}{----------------}@>

Object "/home/sham42/git/twist-mc-test/tests/alloc_lfstack3/test.cpp",
in operator()
  61:           SHOW_NOTE("Alloc(): before");
> <@\textcolor{magenta}{62:           stack.Alloc();}@>
  63:           SHOW_NOTE("Alloc(): after");

Local variables: 
available = 0x0
node_index = 0xffffffff

<@\textcolor{cyan}{------------------------------------------------------------------------------------------}@>
                                         <@\textcolor{cyan}{RUNNING}@>
<@\textcolor{cyan}{------------------------------------------------------------------------------------------}@>

fetch_add: prev = 2

<@\textcolor{cyan}{------------------------------------------------------------------------------------------}@>
                                        <@\textcolor{cyan}{SNAPSHOT 8}@>
<@\textcolor{cyan}{------------------------------------------------------------------------------------------}@>

<@\textcolor{red}{SHARED STATE:}@>

    LFAlloc {
	    LFStack {
		    Head {
			    Blocks: -> C (0x7fc21bc96010) 
		    }

		    Pending {
			    Blocks: -> A (0x7fc21bc96070) -> B (0x7fc21bc96040) 
		    }

		<@\textcolor{red}{[*] allocation\_count\_: 3}@>
	    }

	    Memory {
		    C (0x7fc21bc96010) : 1
		    B (0x7fc21bc96040) : 1
		    A (0x7fc21bc96070) : 1
	    }

    }

<@\textcolor{cyan}{------------------------------------------------------------------------------------------}@>
<@\textcolor{magenta}{
STACK:
}@>
Object "/home/sham42/git/twist-mc-test/tests/alloc_lfstack3/alloc_lfstack.hpp",
in LockFreeStack::Pop()
  42:   Node* Pop() {
> <@\textcolor{magenta}{43:     Node* curr\_top = top\_.load();}@>
  44:     Node* next_top;

Local variables: 
curr_top = 0x100000005
next_top = 0x100000002

<@\textcolor{cyan}{----------------}@>

Object "/home/sham42/git/twist-mc-test/tests/alloc_lfstack3/alloc_lfstack.hpp",
in LFAllocator::Alloc()
> <@\textcolor{magenta}{115:     auto node = item\_list\_.Pop();}@>
  116:     if (node != nullptr) {

Local variables: 
pending_head = 0x7fc21bc96070
node = 0x7ffd4ad1d480

<@\textcolor{cyan}{----------------}@>

Object "/home/sham42/git/twist-mc-test/tests/alloc_lfstack3/test.cpp",
in operator()
  61:           SHOW_NOTE("Alloc(): before");
> <@\textcolor{magenta}{62:           stack.Alloc();}@>
  63:           SHOW_NOTE("Alloc(): after");

Local variables: 
available = 0x0
node_index = 0xffffffff

<@\textcolor{cyan}{------------------------------------------------------------------------------------------}@>
                                         <@\textcolor{cyan}{RUNNING}@>
<@\textcolor{cyan}{------------------------------------------------------------------------------------------}@>

load(): 0x7fc21bc96010

<@\textcolor{cyan}{------------------------------------------------------------------------------------------}@>
                                        <@\textcolor{cyan}{SNAPSHOT 9}@>
<@\textcolor{cyan}{------------------------------------------------------------------------------------------}@>

<@\textcolor{red}{SHARED STATE:}@>

    LFAlloc {
	    LFStack {
		    Head {
			    Blocks: -> C (0x7fc21bc96010) 
		    }

		    Pending {
			    Blocks: -> A (0x7fc21bc96070) -> B (0x7fc21bc96040) 
		    }

		    allocation_count_: 3
	    }

	    Memory {
		    C (0x7fc21bc96010) : 1
		    B (0x7fc21bc96040) : 1
		    A (0x7fc21bc96070) : 1
	    }

    }

<@\textcolor{cyan}{------------------------------------------------------------------------------------------}@>
<@\textcolor{magenta}{
STACK:
}@>
Object "/home/sham42/git/twist-mc-test/tests/alloc_lfstack3/alloc_lfstack.hpp",
in LockFreeStack::Pop()
  48:       }
> <@\textcolor{magenta}{49:       next\_top = curr\_top->next\_.load();}@>
  50:     } while (!top_.compare_exchange_weak(curr_top, next_top));

Local variables: 
curr_top = 0x7fc21bc96010
next_top = 0x100000002

<@\textcolor{cyan}{----------------}@>

Object "/home/sham42/git/twist-mc-test/tests/alloc_lfstack3/alloc_lfstack.hpp",
in LFAllocator::Alloc()
> <@\textcolor{magenta}{115:     auto node = item\_list\_.Pop();}@>
  116:     if (node != nullptr) {

Local variables: 
pending_head = 0x7fc21bc96070
node = 0x7ffd4ad1d480

<@\textcolor{cyan}{----------------}@>

Object "/home/sham42/git/twist-mc-test/tests/alloc_lfstack3/test.cpp",
in operator()
  61:           SHOW_NOTE("Alloc(): before");
> <@\textcolor{magenta}{62:           stack.Alloc();}@>
  63:           SHOW_NOTE("Alloc(): after");

Local variables: 
available = 0x0
node_index = 0xffffffff

<@\textcolor{cyan}{------------------------------------------------------------------------------------------}@>
                                         <@\textcolor{cyan}{RUNNING}@>
<@\textcolor{cyan}{------------------------------------------------------------------------------------------}@>

load(): 0

<@\textcolor{cyan}{------------------------------------------------------------------------------------------}@>
                                       <@\textcolor{cyan}{SNAPSHOT 10}@>
<@\textcolor{cyan}{------------------------------------------------------------------------------------------}@>

<@\textcolor{red}{SHARED STATE:}@>

    LFAlloc {
	    LFStack {
		    Head {
			    Blocks: -> C (0x7fc21bc96010) 
		    }

		    Pending {
			    Blocks: -> A (0x7fc21bc96070) -> B (0x7fc21bc96040) 
		    }

		    allocation_count_: 3
	    }

	    Memory {
		    C (0x7fc21bc96010) : 1
		    B (0x7fc21bc96040) : 1
		    A (0x7fc21bc96070) : 1
	    }

    }

<@\textcolor{cyan}{------------------------------------------------------------------------------------------}@>
<@\textcolor{magenta}{
STACK:
}@>
Object "/home/sham42/git/twist-mc-test/tests/alloc_lfstack3/alloc_lfstack.hpp",
in LockFreeStack::Pop()
  49:       next_top = curr_top->next_.load();
> <@\textcolor{magenta}{50:     \} while (!top\_.compare\_exchange\_weak(curr\_top, next\_top));}@>
  51:     return curr_top;

Local variables: 
curr_top = 0x7fc21bc96010
next_top = 0x0

<@\textcolor{cyan}{----------------}@>

Object "/home/sham42/git/twist-mc-test/tests/alloc_lfstack3/alloc_lfstack.hpp",
in LFAllocator::Alloc()
> <@\textcolor{magenta}{115:     auto node = item\_list\_.Pop();}@>
  116:     if (node != nullptr) {

Local variables: 
pending_head = 0x7fc21bc96070
node = 0x7ffd4ad1d480

<@\textcolor{cyan}{----------------}@>

Object "/home/sham42/git/twist-mc-test/tests/alloc_lfstack3/test.cpp",
in operator()
  61:           SHOW_NOTE("Alloc(): before");
> <@\textcolor{magenta}{62:           stack.Alloc();}@>
  63:           SHOW_NOTE("Alloc(): after");

Local variables: 
available = 0x0
node_index = 0xffffffff

<@\textcolor{cyan}{------------------------------------------------------------------------------------------}@>
                                         <@\textcolor{cyan}{RUNNING}@>
<@\textcolor{cyan}{------------------------------------------------------------------------------------------}@>

compare_exchange_weak(): succeeded = 1

<@\textcolor{cyan}{------------------------------------------------------------------------------------------}@>
                                          <@\textcolor{cyan}{AFTER}@>
<@\textcolor{cyan}{------------------------------------------------------------------------------------------}@>

<@\textcolor{red}{SHARED STATE:}@>

    LFAlloc {
	    LFStack {
		    Head {
			<@\textcolor{red}{[*] Blocks: }@>
		    }

		    Pending {
			    Blocks: -> A (0x7fc21bc96070) -> B (0x7fc21bc96040) 
		    }

		    allocation_count_: 3
	    }

	    Memory {
		<@\textcolor{red}{[*] C (0x7fc21bc96010) : 0}@>
		    B (0x7fc21bc96040) : 1
		    A (0x7fc21bc96070) : 1
	    }

    }

<@\textcolor{cyan}{------------------------------------------------------------------------------------------}@>
<@\textcolor{magenta}{
STACK:
}@>
Object "/home/sham42/git/twist-mc-test/tests/alloc_lfstack3/alloc_lfstack.hpp",
in LFAllocator::Alloc()
> <@\textcolor{magenta}{120:     allocation\_count\_.fetch\_sub(1);}@>

Local variables: 
pending_head = 0x7fc21bc96070
node = 0x7fc21bc96010

<@\textcolor{cyan}{----------------}@>

Object "/home/sham42/git/twist-mc-test/tests/alloc_lfstack3/test.cpp",
in operator()
  61:           SHOW_NOTE("Alloc(): before");
> <@\textcolor{magenta}{62:           stack.Alloc();}@>
  63:           SHOW_NOTE("Alloc(): after");

Local variables: 
available = 0x0
node_index = 0xffffffff

<@\textcolor{yellow}{==========================================================================================}@>
                                    <@\textcolor{yellow}{RUN 14 (THREAD 3)}@>
<@\textcolor{yellow}{==========================================================================================}@>

<@\textcolor{cyan}{------------------------------------------------------------------------------------------}@>
                                          <@\textcolor{cyan}{BEFORE}@>
<@\textcolor{cyan}{------------------------------------------------------------------------------------------}@>
<@\textcolor{magenta}{
STACK:
}@>
Object "/home/sham42/git/twist-mc-test/tests/alloc_lfstack3/alloc_lfstack.hpp",
in LockFreeStack::TryClear(Node*)
  55:     Node* desired = nullptr;
> <@\textcolor{magenta}{56:     return top\_.compare\_exchange\_strong(expected, desired);}@>
  57:   }

Local variables: 
desired = 0x0

<@\textcolor{cyan}{----------------}@>

Object "/home/sham42/git/twist-mc-test/tests/alloc_lfstack3/alloc_lfstack.hpp",
in LFAllocator::Alloc()
  100:       if (pending_head != nullptr /*&& keep_count == pending_count_.load()*/
> <@\textcolor{magenta}{101:           \&\& pending\_list\_.TryClear(pending\_head)) \{}@>
  102:         auto head = pending_head;

Local variables: 
pending_head = 0x7fc21bc96070
node = 0x7ffd4ad1d480

<@\textcolor{cyan}{----------------}@>

Object "/home/sham42/git/twist-mc-test/tests/alloc_lfstack3/test.cpp",
in operator()
  61:           SHOW_NOTE("Alloc(): before");
> <@\textcolor{magenta}{62:           stack.Alloc();}@>
  63:           SHOW_NOTE("Alloc(): after");

Local variables: 
available = 0x0
node_index = 0xffffffff

<@\textcolor{cyan}{------------------------------------------------------------------------------------------}@>
                                         <@\textcolor{cyan}{RUNNING}@>
<@\textcolor{cyan}{------------------------------------------------------------------------------------------}@>

compare_exchange_strong(): succeeded = 1

<@\textcolor{cyan}{------------------------------------------------------------------------------------------}@>
                                        <@\textcolor{cyan}{SNAPSHOT 1}@>
<@\textcolor{cyan}{------------------------------------------------------------------------------------------}@>

<@\textcolor{red}{SHARED STATE:}@>

    LFAlloc {
	    LFStack {
		    Head {
			    Blocks: 
		    }

		    Pending {
			<@\textcolor{red}{[*] Blocks: }@>
		    }

		    allocation_count_: 3
	    }

	    Memory {
		    C (0x7fc21bc96010) : 0
		    B (0x7fc21bc96040) : 1
		    A (0x7fc21bc96070) : 1
	    }

    }

<@\textcolor{cyan}{------------------------------------------------------------------------------------------}@>
<@\textcolor{magenta}{
STACK:
}@>
Object "/home/sham42/git/twist-mc-test/tests/alloc_lfstack3/alloc_lfstack.hpp",
in LFAllocator::Alloc()
> <@\textcolor{magenta}{104:         pending\_head = pending\_head->next\_.load();}@>
  105:         head->Release();

Local variables: 
head = 0x7fc21bc96070
pending_head = 0x7fc21bc96070
node = 0x7ffd4ad1d480

<@\textcolor{cyan}{----------------}@>

Object "/home/sham42/git/twist-mc-test/tests/alloc_lfstack3/test.cpp",
in operator()
  61:           SHOW_NOTE("Alloc(): before");
> <@\textcolor{magenta}{62:           stack.Alloc();}@>
  63:           SHOW_NOTE("Alloc(): after");

Local variables: 
available = 0x0
node_index = 0xffffffff

<@\textcolor{cyan}{------------------------------------------------------------------------------------------}@>
                                         <@\textcolor{cyan}{RUNNING}@>
<@\textcolor{cyan}{------------------------------------------------------------------------------------------}@>

load(): 0x7fc21bc96040

<@\textcolor{cyan}{------------------------------------------------------------------------------------------}@>
                                        <@\textcolor{cyan}{SNAPSHOT 2}@>
<@\textcolor{cyan}{------------------------------------------------------------------------------------------}@>

<@\textcolor{red}{SHARED STATE:}@>

    LFAlloc {
	    LFStack {
		    Head {
			    Blocks: 
		    }

		    Pending {
			    Blocks: 
		    }

		    allocation_count_: 3
	    }

	    Memory {
		    C (0x7fc21bc96010) : 0
		    B (0x7fc21bc96040) : 1
		<@\textcolor{red}{[*] A (0x7fc21bc96070) : 0}@>
	    }

    }

<@\textcolor{cyan}{------------------------------------------------------------------------------------------}@>
<@\textcolor{magenta}{
STACK:
}@>
Object "/home/sham42/git/twist-mc-test/tests/alloc_lfstack3/alloc_lfstack.hpp",
in LockFreeStack::Append(Node*)
  25:     Node* tail = head;
> <@\textcolor{magenta}{26:     Node* next = tail->next\_.load();}@>
  27:     while (next != nullptr) {

Local variables: 
tail = 0x7fc21bc96040
next = 0x7fc21baaa6a0
prev_top = 0x7fc21baaa690

<@\textcolor{cyan}{----------------}@>

Object "/home/sham42/git/twist-mc-test/tests/alloc_lfstack3/alloc_lfstack.hpp",
in LFAllocator::Alloc()
> <@\textcolor{magenta}{107:         item\_list\_.Append(pending\_head);}@>
  108:         // pending_count_.fetch_add(1);

Local variables: 
head = 0x7fc21bc96070
pending_head = 0x7fc21bc96040
node = 0x7ffd4ad1d480

<@\textcolor{cyan}{----------------}@>

Object "/home/sham42/git/twist-mc-test/tests/alloc_lfstack3/test.cpp",
in operator()
  61:           SHOW_NOTE("Alloc(): before");
> <@\textcolor{magenta}{62:           stack.Alloc();}@>
  63:           SHOW_NOTE("Alloc(): after");

Local variables: 
available = 0x0
node_index = 0xffffffff

<@\textcolor{cyan}{------------------------------------------------------------------------------------------}@>
                                         <@\textcolor{cyan}{RUNNING}@>
<@\textcolor{cyan}{------------------------------------------------------------------------------------------}@>

load(): 0

<@\textcolor{cyan}{------------------------------------------------------------------------------------------}@>
                                        <@\textcolor{cyan}{SNAPSHOT 3}@>
<@\textcolor{cyan}{------------------------------------------------------------------------------------------}@>

<@\textcolor{red}{SHARED STATE:}@>

    LFAlloc {
	    LFStack {
		    Head {
			    Blocks: 
		    }

		    Pending {
			    Blocks: 
		    }

		    allocation_count_: 3
	    }

	    Memory {
		    C (0x7fc21bc96010) : 0
		    B (0x7fc21bc96040) : 1
		    A (0x7fc21bc96070) : 0
	    }

    }

<@\textcolor{cyan}{------------------------------------------------------------------------------------------}@>
<@\textcolor{magenta}{
STACK:
}@>
Object "/home/sham42/git/twist-mc-test/tests/alloc_lfstack3/alloc_lfstack.hpp",
in LockFreeStack::Append(Node*)
> <@\textcolor{magenta}{32:     Node* prev\_top = top\_.load();}@>
  33:     do {

Local variables: 
tail = 0x7fc21bc96040
next = 0x0
prev_top = 0x7fc21baaa690

<@\textcolor{cyan}{----------------}@>

Object "/home/sham42/git/twist-mc-test/tests/alloc_lfstack3/alloc_lfstack.hpp",
in LFAllocator::Alloc()
> <@\textcolor{magenta}{107:         item\_list\_.Append(pending\_head);}@>
  108:         // pending_count_.fetch_add(1);

Local variables: 
head = 0x7fc21bc96070
pending_head = 0x7fc21bc96040
node = 0x7ffd4ad1d480

<@\textcolor{cyan}{----------------}@>

Object "/home/sham42/git/twist-mc-test/tests/alloc_lfstack3/test.cpp",
in operator()
  61:           SHOW_NOTE("Alloc(): before");
> <@\textcolor{magenta}{62:           stack.Alloc();}@>
  63:           SHOW_NOTE("Alloc(): after");

Local variables: 
available = 0x0
node_index = 0xffffffff

<@\textcolor{cyan}{------------------------------------------------------------------------------------------}@>
                                         <@\textcolor{cyan}{RUNNING}@>
<@\textcolor{cyan}{------------------------------------------------------------------------------------------}@>

load(): 0

<@\textcolor{cyan}{------------------------------------------------------------------------------------------}@>
                                        <@\textcolor{cyan}{SNAPSHOT 4}@>
<@\textcolor{cyan}{------------------------------------------------------------------------------------------}@>

<@\textcolor{red}{SHARED STATE:}@>

    LFAlloc {
	    LFStack {
		    Head {
			    Blocks: 
		    }

		    Pending {
			    Blocks: 
		    }

		    allocation_count_: 3
	    }

	    Memory {
		    C (0x7fc21bc96010) : 0
		    B (0x7fc21bc96040) : 1
		    A (0x7fc21bc96070) : 0
	    }

    }

<@\textcolor{cyan}{------------------------------------------------------------------------------------------}@>
<@\textcolor{magenta}{
STACK:
}@>
Object "/home/sham42/git/twist-mc-test/tests/alloc_lfstack3/alloc_lfstack.hpp",
in LockFreeStack::Append(Node*)
  33:     do {
> <@\textcolor{magenta}{34:       tail->next\_ = prev\_top;}@>
  35:     } while (!top_.compare_exchange_weak(prev_top, head));

Local variables: 
tail = 0x7fc21bc96040
next = 0x0
prev_top = 0x0

<@\textcolor{cyan}{----------------}@>

Object "/home/sham42/git/twist-mc-test/tests/alloc_lfstack3/alloc_lfstack.hpp",
in LFAllocator::Alloc()
> <@\textcolor{magenta}{107:         item\_list\_.Append(pending\_head);}@>
  108:         // pending_count_.fetch_add(1);

Local variables: 
head = 0x7fc21bc96070
pending_head = 0x7fc21bc96040
node = 0x7ffd4ad1d480

<@\textcolor{cyan}{----------------}@>

Object "/home/sham42/git/twist-mc-test/tests/alloc_lfstack3/test.cpp",
in operator()
  61:           SHOW_NOTE("Alloc(): before");
> <@\textcolor{magenta}{62:           stack.Alloc();}@>
  63:           SHOW_NOTE("Alloc(): after");

Local variables: 
available = 0x0
node_index = 0xffffffff

<@\textcolor{cyan}{------------------------------------------------------------------------------------------}@>
                                        <@\textcolor{cyan}{SNAPSHOT 5}@>
<@\textcolor{cyan}{------------------------------------------------------------------------------------------}@>

<@\textcolor{red}{SHARED STATE:}@>

    LFAlloc {
	    LFStack {
		    Head {
			    Blocks: 
		    }

		    Pending {
			    Blocks: 
		    }

		    allocation_count_: 3
	    }

	    Memory {
		    C (0x7fc21bc96010) : 0
		    B (0x7fc21bc96040) : 1
		    A (0x7fc21bc96070) : 0
	    }

    }

<@\textcolor{cyan}{------------------------------------------------------------------------------------------}@>
<@\textcolor{magenta}{
STACK:
}@>
Object "/home/sham42/git/twist-mc-test/tests/alloc_lfstack3/alloc_lfstack.hpp",
in LockFreeStack::Append(Node*)
  34:       tail->next_ = prev_top;
> <@\textcolor{magenta}{35:     \} while (!top\_.compare\_exchange\_weak(prev\_top, head));}@>
  36:   }

Local variables: 
tail = 0x7fc21bc96040
next = 0x0
prev_top = 0x0

<@\textcolor{cyan}{----------------}@>

Object "/home/sham42/git/twist-mc-test/tests/alloc_lfstack3/alloc_lfstack.hpp",
in LFAllocator::Alloc()
> <@\textcolor{magenta}{107:         item\_list\_.Append(pending\_head);}@>
  108:         // pending_count_.fetch_add(1);

Local variables: 
head = 0x7fc21bc96070
pending_head = 0x7fc21bc96040
node = 0x7ffd4ad1d480

<@\textcolor{cyan}{----------------}@>

Object "/home/sham42/git/twist-mc-test/tests/alloc_lfstack3/test.cpp",
in operator()
  61:           SHOW_NOTE("Alloc(): before");
> <@\textcolor{magenta}{62:           stack.Alloc();}@>
  63:           SHOW_NOTE("Alloc(): after");

Local variables: 
available = 0x0
node_index = 0xffffffff

<@\textcolor{cyan}{------------------------------------------------------------------------------------------}@>
                                         <@\textcolor{cyan}{RUNNING}@>
<@\textcolor{cyan}{------------------------------------------------------------------------------------------}@>

compare_exchange_weak(): succeeded = 1

<@\textcolor{cyan}{------------------------------------------------------------------------------------------}@>
                                          <@\textcolor{cyan}{AFTER}@>
<@\textcolor{cyan}{------------------------------------------------------------------------------------------}@>

<@\textcolor{red}{SHARED STATE:}@>

    LFAlloc {
	    LFStack {
		    Head {
			<@\textcolor{red}{[*] Blocks: -> B (0x7fc21bc96040) }@>
		    }

		    Pending {
			    Blocks: 
		    }

		    allocation_count_: 3
	    }

	    Memory {
		    C (0x7fc21bc96010) : 0
		    B (0x7fc21bc96040) : 1
		    A (0x7fc21bc96070) : 0
	    }

    }

<@\textcolor{cyan}{------------------------------------------------------------------------------------------}@>
<@\textcolor{magenta}{
STACK:
}@>
Object "/home/sham42/git/twist-mc-test/tests/alloc_lfstack3/alloc_lfstack.hpp",
in LFAllocator::Alloc()
  108:         // pending_count_.fetch_add(1);
> <@\textcolor{magenta}{109:         allocation\_count\_.fetch\_sub(1);}@>

Local variables: 
head = 0x7fc21bc96070
pending_head = 0x7fc21bc96040
node = 0x7ffd4ad1d480

<@\textcolor{cyan}{----------------}@>

Object "/home/sham42/git/twist-mc-test/tests/alloc_lfstack3/test.cpp",
in operator()
  61:           SHOW_NOTE("Alloc(): before");
> <@\textcolor{magenta}{62:           stack.Alloc();}@>
  63:           SHOW_NOTE("Alloc(): after");

Local variables: 
available = 0x0
node_index = 0xffffffff

<@\textcolor{yellow}{==========================================================================================}@>
                                    <@\textcolor{yellow}{RUN 15 (THREAD 2)}@>
<@\textcolor{yellow}{==========================================================================================}@>

<@\textcolor{cyan}{------------------------------------------------------------------------------------------}@>
                                          <@\textcolor{cyan}{BEFORE}@>
<@\textcolor{cyan}{------------------------------------------------------------------------------------------}@>
<@\textcolor{magenta}{
STACK:
}@>
Object "/home/sham42/git/twist-mc-test/tests/alloc_lfstack3/alloc_lfstack.hpp",
in LockFreeStack::Pop()
  49:       next_top = curr_top->next_.load();
> <@\textcolor{magenta}{50:     \} while (!top\_.compare\_exchange\_weak(curr\_top, next\_top));}@>
  51:     return curr_top;

Local variables: 
curr_top = 0x7fc21bc96040
next_top = 0x7fc21bc96010

<@\textcolor{cyan}{----------------}@>

Object "/home/sham42/git/twist-mc-test/tests/alloc_lfstack3/alloc_lfstack.hpp",
in LFAllocator::Alloc()
> <@\textcolor{magenta}{115:     auto node = item\_list\_.Pop();}@>
  116:     if (node != nullptr) {

Local variables: 
pending_head = 0x0
node = 0x7ffd4ad1d480

<@\textcolor{cyan}{----------------}@>

Object "/home/sham42/git/twist-mc-test/tests/alloc_lfstack3/test.cpp",
in operator()
  61:           SHOW_NOTE("Alloc(): before");
> <@\textcolor{magenta}{62:           stack.Alloc();}@>
  63:           SHOW_NOTE("Alloc(): after");

Local variables: 
available = 0x1
node_index = 0xffffffff

<@\textcolor{cyan}{------------------------------------------------------------------------------------------}@>
                                         <@\textcolor{cyan}{RUNNING}@>
<@\textcolor{cyan}{------------------------------------------------------------------------------------------}@>

compare_exchange_weak(): succeeded = 1

<@\textcolor{cyan}{------------------------------------------------------------------------------------------}@>
                                          <@\textcolor{cyan}{AFTER}@>
<@\textcolor{cyan}{------------------------------------------------------------------------------------------}@>

<@\textcolor{red}{SHARED STATE:}@>

    LFAlloc {
	    LFStack {
		    Head {
			<@\textcolor{red}{[*] Blocks: -> C (0x7fc21bc96010) }@>
		    }

		    Pending {
			    Blocks: 
		    }

		    allocation_count_: 3
	    }

	    Memory {
		    C (0x7fc21bc96010) : 0
		<@\textcolor{red}{[*] B (0x7fc21bc96040) : 0}@>
		    A (0x7fc21bc96070) : 0
	    }

    }

<@\textcolor{cyan}{------------------------------------------------------------------------------------------}@>
<@\textcolor{magenta}{
STACK:
}@>
Object "/home/sham42/git/twist-mc-test/tests/alloc_lfstack3/alloc_lfstack.hpp",
in LFAllocator::Alloc()
> <@\textcolor{magenta}{120:     allocation\_count\_.fetch\_sub(1);}@>

Local variables: 
pending_head = 0x0
node = 0x7fc21bc96040

<@\textcolor{cyan}{----------------}@>

Object "/home/sham42/git/twist-mc-test/tests/alloc_lfstack3/test.cpp",
in operator()
  61:           SHOW_NOTE("Alloc(): before");
> <@\textcolor{magenta}{62:           stack.Alloc();}@>
  63:           SHOW_NOTE("Alloc(): after");

Local variables: 
available = 0x1
node_index = 0xffffffff

<@\textcolor{red}{==========================================================================================}@>
                                     <@\textcolor{red}{ASSERTION FAILED}@>
<@\textcolor{red}{==========================================================================================}@>

ItemList: assertion failed
<@\textcolor{red}{
==========================================================================================
}@>

\end{lstlisting}

\end{multicols*}

\end{allintypewriter}

\end{adjustwidth}


        % Приложения

\end{document}
