\chapter*{Аннотация}

Исполнение многопоточной программы по своей природе недетерминированно, а значит стандартных методик тестирования для верификации многопоточного кода недостаточно, требуются специальные техники: fault injection и model checking.

Целью данной работы было изучить основные методы верификации конкурентного кода, выявить их недостатки и реализовать model checker, который совместил бы преимущества обоих подходов: (как fault injection) проверял бы не модель, а код на \CC, и (как model checking) перебирал бы все достижимые из теста состояния исполнения.

В работе мы опишем реализацию разработанного model checker-а и продемонстрируем его работоспособность на двух нетривиальных примерах: lock-free аллокаторе и фреймворке экзекьюторов.
