\chapter{Заключение}



\section{Результаты}

Реализован model checker, который для многопоточного теста на C++ проверяет заданный инвариант во всех возможных состояниях, достижимых при исполнении этого теста. При нарушении инварианта model checker печатает кратчайшую детализированную траекторию, которая приводит к этому.

Model checker протестирован на нетривиальных примерах: он находит сложные баги из десятков шагов (ABA в lock-free стеке) и способен проверять сложный составной код (фреймворк экзекьюторов). 

Реализованы и описаны инструменты  для управления перебором в model checker-е: \mintinline{c++}{ForkGuarded<T>} для управления гранулярностью атомарности, \mintinline{c++}{Either} / \mintinline{c++}{Random} для выражения недетерминированного ветвления, \mintinline{c++}{Prune} для отсечения перебора. 

\section{Направления для дальнейшего исследования}

\begin{itemize}

\item	Поддержать слабые модели памяти.

\item	Автоматически инструментировать тестируемый код.

\end{itemize}

